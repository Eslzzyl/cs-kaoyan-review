% draft会跳过文档中的所有图片。正式导出时需要删掉draft参数。
\documentclass[12pt, a4paper, oneside]{ctexart}

\usepackage{amsmath}
\usepackage{amssymb}
\usepackage{amsopn}
\usepackage{bm}
\usepackage{graphicx}
\usepackage{mathrsfs}
\usepackage{geometry}
\usepackage{framed}
\usepackage{color}
\usepackage{caption}
\usepackage{listings}
\usepackage{fancyhdr}
\usepackage{booktabs}
\usepackage{makecell}
\usepackage{indentfirst}
\usepackage{authblk}
\usepackage{multicol}
% \usepackage{draftwatermark}       % 需要应用水印时取消注释
\usepackage{enumitem}
\usepackage[hidelinks]{hyperref}
\usepackage{tikz}
\usepackage{ulem}
\usetikzlibrary{positioning, shapes.geometric}

% 分栏线宽
\columnseprule=0.4pt

% 定制第二级无序列表的点样式
\setlist[itemize,2]{label=$\diamond$}

% 页边距
\geometry{a4paper, scale=0.8}

\pagestyle{fancy}

% 调整页眉高度,用于去除警告
\setlength{\headheight}{25pt}

\fancyhf{}      % 清空页眉页脚设置
\fancyhead[L] {
    % 工大计算机系logo
    \includegraphics[height=7mm]{./images/logo1.jpg}
}
\fancyhead[C]{《数据结构》复习}
\fancyhead[R]{\leftmark}    % 右侧页眉:当前章标题

% 页脚居中放置页码
\fancyfoot[C]{\thepage}

% 设置章节标题自动编号的格式
\ctexset{
  section/number=\chinese{section},
%   subsection/name={,},
%   subsection/number=\chinese{subsection}
}

% 行距。ctexart默认值为1.3
\linespread{1.2}

\lstset{
  language=c,
  basicstyle=\ttfamily,
  frame=single,
  numbers=left
}

% \SetWatermarkText{Eslzzyl整理}            % 设置水印内容
% \SetWatermarkLightness{0.9}             % 设置水印透明度 0-1
% \SetWatermarkScale{0.8}                   % 设置水印大小 0-1

\renewcommand{\headrulewidth}{1pt}  %页眉线宽,设为0可以去页眉线
\renewcommand{\footrulewidth}{1pt}  %脚注线的宽度

\definecolor{shadecolor}{RGB}{241, 241, 255}

\title{
    \includegraphics[width=0.3\textwidth]{images/hfut-badge.pdf}
    
    \vspace{20pt}
    《数据结构》总复习
}
\author{Eslzzyl}
\date{\today}

\newcounter{problemname}
\newenvironment{problem}{\begin{shaded}\stepcounter{problemname}\par\noindent\textbf{例题\arabic{problemname}. }}{\end{shaded}\par}
\newenvironment{solution}{\begin{shaded}\par\noindent\textbf{解答:}}{\end{shaded}\par}
% \newenvironment{solution}{\par\noindent\textbf{答案. }}{\par}
% \newenvironment{note}{\par\noindent\textbf{例题\arabic{problemname}的注记. }}{\\\par}
\newenvironment{note}{\par\noindent\textbf{注记. }}{\par}

\begin{document}

\maketitle
\newpage
\tableofcontents
\vspace{20pt}
% 如果在目录处有备注,可以写在这里。

\newpage

\section{绪论}

\subsection{数据结构的基本概念}

\subsubsection{基本概念和术语}

\begin{itemize}
  \item {\bf 数据},是信息的载体,是计算机程序加工的原料。
  \item {\bf 数据元素},是数据的基本单位,如一个学生记录。可由若干个数据项组成,如包含学号、姓名、性别等数据项。
  \item {\bf 数据对象},是具有\textbf{相同性质}的数据元素的集合。
  \item {\bf 数据类型},是一个值的集合和定义在此集合上的一组操作的总称。
  \begin{itemize}
    \item 原子类型:不可再分的数据类型。
    \item 结构类型:可以再分的数据类型。
    \item 抽象数据类型
  \end{itemize}
  \item {\bf 数据结构},是相互之间存在一种或多种特定关系的数据元素的集合。数据结构包括三方面内容:
  \begin{itemize}
    \item 逻辑结构,决定算法的设计
    \item 存储结构,又叫映像,决定算法的实现
    \item 数据的运算
  \end{itemize}
\end{itemize}

\subsubsection{数据结构三要素}

\begin{enumerate}
  \item {\bf 数据的逻辑结构}
  
  \begin{figure}[h]
    \centering
    \includegraphics[width=0.7\textwidth]{./images/logical-structure.png}
    \caption{数据的逻辑结构}
    \label{logical-structure}
  \end{figure}
  
  逻辑结构可分为线性结构和非线性结构。见图\ref{logical-structure}。

  \begin{itemize}
    \item {\bf 集合}:结构中的元素之间除了“同属一个集合”之外,没有别的关系。
    \item {\bf 线性结构}:元素之间存在一对一的关系。
    \item {\bf 树形结构}:元素之间存在一对多的关系。
    \item {\bf 图状结构或网状结构}:元素之间存在多对多的关系。
  \end{itemize}

  \item {\bf 数据的存储结构}
  
  存储结构是用计算机语言实现的逻辑结构。
  \begin{itemize}
    \item {\bf 顺序存储}
    \item {\bf 链式存储}
    \item {\bf 索引存储}
    \item {\bf 散列存储}
  \end{itemize}

  \item {\bf 数据的运算}
\end{enumerate}

\subsection{算法和算法评价}

\subsubsection{算法的基本概念}

算法(Algorithm)是对特定问题求解步骤的一种描述。有以下5个重要特性:
\begin{enumerate}
  \item {\bf 有穷性},算法必须在有限个步骤之后结束,且每一步必须消耗有限的时间。
  \item {\bf 确定性},同一个算法对同一个输入必定得出相同的输出。
  \begin{itemize}
    \item 随机类算法不违反确定性,因为计算机是伪随机,对于同一个随机种子,生成的随机数序列总是一致的。
  \end{itemize}
  \item {\bf 可行性}
  \item {\bf 输入},算法有0个或更多个输入。
  \item {\bf 输出},算法有一个或更多个输出。
\end{enumerate}

设计一个好的算法应该考虑:
\begin{itemize}
  \item 正确性
  \item 可读性
  \item 健壮性
  \item 高效率与低存储量需求
\end{itemize}

\subsubsection{算法效率的度量}

\begin{enumerate}
  \item {\bf 时间复杂度}
  
  通常采用算法中基本运算的频度来分析时间复杂度。

  一般总是考虑在最坏情况下的时间复杂度。

  常见的渐进时间复杂度:
  \begin{equation*}
    O(1)<O(\log_2 n)<O(n)<O(n\log_2 n)<O(n^2)<O(n^3)<O(n!)<O(n^n)
  \end{equation*}
  \item {\bf 空间复杂度}
  
  算法\textbf{原地工作}是指算法所需的辅助空间为常量,即$O(1)$。
\end{enumerate}

\section{线性表}

\subsection{线性表的定义和基本操作}

\subsubsection{线性表的定义}

线性表是具有相同数据类型的$n$($n\geq 0$)个数据元素的有限序列,其中$n$为表长。
\begin{equation*}
  L=(a_1,a_2,\cdots,a_i,a_{i+1},\cdots,a_n)
\end{equation*}

线性表的特点:
\begin{itemize}
  \item 表中元素的个数有限。
  \item 元素具有逻辑上的顺序性。
  \item 元素都是数据元素。
  \item 元素的数据类型都相同,占有相同大小的存储空间。
  \item 元素具有抽象性,不关心元素中保存的是什么内容。
\end{itemize}

线性表是逻辑结构,顺序表和链表是物理结构。

\subsubsection{线性表的基本操作}

\begin{itemize}
  \item \verb|InitList(&L)|:初始化表,构造一个空的线性表。
  \item \verb|Length(L)|:返回表的长度
  \item \verb|LocateElem(L, e)|:按值查找
  \item \verb|GetElem(L, i)|,按位置查找
  \item \verb|ListInsert(&L, i, e)|:插入
  \item \verb|ListDelete(&L, i, &e)|:删除
  \item \verb|PrintList(L)|:输出表
  \item \verb|Empty(L)|:判空,空返回\verb|true|
  \item \verb|DestroyList(&L)|:销毁表
\end{itemize}

\subsection{线性表的顺序表示}

注意线性表的下标是从1开始算,数组的下标是从0开始算。

时间复杂度分析(平均):
\begin{itemize}
  \item 随机访问:$O(1)$
  \item 插入:$O(n)$
  \item 删除:$O(n)$
  \item 顺序查找:$O(n)$
\end{itemize}

\subsection{线性表的链式表示}

链表往往引入头结点,有如下好处:
\begin{itemize}
  \item 在链表的第一个位置上的操作和其他位置一致,无需进行特殊处理。
  \item 无论链表是否为空,头指针都指向一个节点,这样空表和非空表的处理得到了统一。
\end{itemize}

建立新链表:
\begin{itemize}
  \item 头插:在头部插入节点,形成的链表的顺序是反的。
  \item 尾插:在尾部插入节点,顺序正常,但需要添加额外的尾指针。
\end{itemize}

时间复杂度分析:
\begin{itemize}
  \item 建立新链表(头插/尾插):$O(n)$
  \item 按序号查找:$O(n)$
  \item 按值查找:$O(n)$
  \item 按值删除:主要耗费在查找上。$O(n)$
  \item 按值插入:同上
  \item 求表长:$O(n)$
\end{itemize}

插入元素时,往往要先找到元素所在的位置,然后执行后插。查找的复杂度是$O(n)$,后插的复杂度仅为$O(1)$。因此:
\begin{itemize}
  \item 当给出插入位置的值时,总复杂度为$O(n)$;
  \item 当给出插入位置的地址时,总复杂度为$O(1)$。
\end{itemize}

有时需要前插,此时要找到目标结点的前面一个节点。对于单链表,必须按照顺序向后查找。此时无论给出的是值还是地址,总复杂度都为$O(n)$;但有一种优化方法:在目标结点的后继插入,然后交换目标结点和新插入结点的值。此时已知地址时,总复杂度为$O(1)$。

类似地,删除结点时,必须得知前驱结点,这样才能正确设置链接关系。但如果直接将后继结点的值赋予待删除结点,然后删除后继结点,这样也能有$O(1)$时间复杂度。

\subsubsection{双链表}

\subsubsection{循环链表}

循环单链表的判空条件不是头结点的指针是否为空,而是它是否等于头指针。

循环单链表少用头指针而多用尾指针。因为尾指针的\verb|next|就是头结点。

\subsubsection{静态链表}

静态链表是用数组实现的链表。不太方便,但是适合没有指针的语言。

\subsubsection{顺序表的链表的比较}

\begin{itemize}
  \item {\bf 存取(读写)方式}
  
  顺序表既能顺序读取,又能随机读取;链表只能顺序读取。

  \item {\bf 逻辑结构与物理结构}
  
  逻辑上都是线性表,但顺序表的结点物理上也是连续的,链表的结点靠指针维持连续性。

  \item {\bf 查找、插入和删除操作}
  \begin{itemize}
    \item 对于按值查找,若表无序,则二者均为$O(n)$;若有序,则顺序表可用二分查找,$O(\log_2 n)$。
    \item 对于按序号查找,顺序表可以随机访问,$O(1)$;链表则为$O(n)$。
  \end{itemize}

  对于插入和删除操作,顺序表的平均复杂度为$O(n)$,链表为$O(1)$。

  \item {\bf 空间分配}
  
  顺序表的分配比较繁琐,链表的分配灵活高效。
\end{itemize}

\section{栈、队列和数组}

本章多为选择题。

\subsection{栈}

\subsubsection{栈的基本概念}



\end{document}