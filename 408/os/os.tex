% draft会跳过文档中的所有图片。正式导出时需要删掉draft参数。
\documentclass[12pt, a4paper, oneside]{ctexart}

\usepackage{amsmath}
\usepackage{amssymb}
\usepackage{bm}
\usepackage{graphicx}
\usepackage{mathrsfs}
\usepackage{geometry}
\usepackage{framed}
\usepackage{color}
\usepackage{caption}
\usepackage{listings}
\usepackage{fancyhdr}
\usepackage{booktabs}
\usepackage{makecell}
\usepackage{indentfirst}
\usepackage{authblk}
\usepackage{multicol}
% \usepackage{draftwatermark}       % 需要应用水印时取消注释
\usepackage{enumitem}
\usepackage[hidelinks]{hyperref}
\usepackage{tikz}
\usepackage{ulem}
\usetikzlibrary{positioning, shapes.geometric}

% 分栏线宽
\columnseprule=0.4pt

% 定制第二级无序列表的点样式
\setlist[itemize,2]{label=$\diamond$}

\pagestyle{fancy}

\fancyhf{}      % 清空页眉页脚设置
\fancyhead[L] {
    % 工大计算机系logo
    \includegraphics[height=7mm]{./images/logo1.jpg}
}
\fancyhead[C]{《操作系统》复习}
\fancyhead[R]{\leftmark}    % 右侧页眉:当前章标题

% 页脚居中放置页码
\fancyfoot[C]{\thepage}

% 设置章节标题自动编号的格式
\ctexset{
  section/number=\chinese{section},
%   subsection/name={,},
%   subsection/number=\chinese{subsection}
}

% 行距。ctexart默认值为1.3
\linespread{1.2}

\lstset{
  language=pascal,
  basicstyle=\ttfamily,
  frame=single,
  numbers=left
}

% \SetWatermarkText{Eslzzyl整理}            % 设置水印内容
% \SetWatermarkLightness{0.9}             % 设置水印透明度 0-1
% \SetWatermarkScale{0.8}                   % 设置水印大小 0-1

\renewcommand{\headrulewidth}{1pt}  %页眉线宽,设为0可以去页眉线
\renewcommand{\footrulewidth}{1pt}  %脚注线的宽度

\definecolor{shadecolor}{RGB}{241, 241, 255}

\title{
    \includegraphics[width=0.3\textwidth]{images/hfut-badge.pdf}
    
    \vspace{20pt}
    《操作系统》总复习
}
\author{Eslzzyl}
\date{\today}

\newcounter{problemname}
\newenvironment{problem}{\begin{shaded}\stepcounter{problemname}\par\noindent\textbf{例题\arabic{problemname}. }}{\end{shaded}\par}
\newenvironment{solution}{\begin{shaded}\par\noindent\textbf{解答:}}{\end{shaded}\par}
% \newenvironment{solution}{\par\noindent\textbf{答案. }}{\par}
% \newenvironment{note}{\par\noindent\textbf{例题\arabic{problemname}的注记. }}{\\\par}
\newenvironment{note}{\par\noindent\textbf{注记. }}{\par}

\begin{document}

\maketitle
\newpage
\tableofcontents
\vspace{20pt}
% 如果在目录处有备注,可以写在这里。

\newpage

\section{计算机系统概述}

本章主要是选择题。

\subsection{操作系统的基本概念}

\subsubsection{操作系统的概念}

计算机系统自上而下可分为4层:
\begin{enumerate}
  \item 用户
  \item 应用程序
  \item 操作系统
  \item 硬件
\end{enumerate}

操作系统的\textbf{定义}:操作系统是一组管理和控制计算机软件和硬件资源,合理组织计算机系统工作流程,以及方便用户使用的\textbf{程序的集合}。

OS是计算机系统中最基本的系统软件。

\subsubsection{操作系统的特征}

四大基本特征:
\begin{itemize}
  \item {\bf 并发(Concurrence)}
  \begin{itemize}
    \item 并发:多个事件在同一时间间隔内发生。
    \item 并行:多个事件在同一时刻发生。
  \end{itemize}
  操作系统的并发性是通过分时得以实现的。
  \item {\bf 共享(Sharing)}
  \begin{itemize}
    \item 互斥共享方式
    \item 同时访问方式
  \end{itemize}
  \item {\bf 虚拟(Virtual)}
  
  虚拟技术有两种实现方式:
  \begin{itemize}
    \item 时分复用技术,如处理器的分时共享;
    \item 空分复用技术,如虚拟存储器。
  \end{itemize}
  \item {\bf 异步(Asynchronism)}
  
  多道程序环境下,程序的执行是走走停停的,即以不可知的速度向前推进,这就是进程的异步性。OS必须保证在相同的运行环境下,进程多次运行的结果是一致的。
\end{itemize}

\subsubsection{操作系统的目标和功能}

\begin{enumerate}
  \item 系统资源的\textbf{管理}者
  \begin{itemize}
    \item 处理机管理,也就是对进程的管理,包括
    \begin{itemize}
      \item 进程控制
      \item 进程同步
      \item 进程通信
      \item 死锁处理
      \item 处理机调度
    \end{itemize}
    \item 存储器管理
    \begin{itemize}
      \item 内存分配与回收
      \item 地址映射
      \item 内存保护和共享
      \item 内存扩充
    \end{itemize}
    \item 文件管理
    \begin{itemize}
      \item 文件存储空间的管理
      \item 目录管理
      \item 文件读写管理和保护
    \end{itemize}
    \item 设备管理
    \begin{itemize}
      \item 缓冲管理
      \item 设备分配
      \item 设备处理
      \item 虚拟设备
    \end{itemize}
  \end{itemize}
  \item 作为用户和硬件系统之间的\textbf{接口}
  \begin{itemize}
    \item 命令接口
    \begin{itemize}
      \item 联机方式:交互式
      \begin{itemize}
        \item 字符式
        \item 图形用户接口GUI
      \end{itemize}
      \item 脱机方式:批处理式
    \end{itemize}
    \item 程序接口:由一组系统调用组成
  \end{itemize}
  \item 实现对计算机资源的\textbf{扩充}
\end{enumerate}

\subsection{操作系统发展历程}

推动OS发展的主要动力:
\begin{itemize}
  \item 不断提高计算机资源利用率
  \item 方便用户
  \item 器件的不断更新换代
  \item 计算机体系结构的发展
  \item 不断提出新的应用需求
\end{itemize}

\subsubsection{无操作系统时代}

对计算机的所有操作采用人工操作方式完成。人工操作方式的缺点:
\begin{itemize}
  \item 用户独占全机
  \item CPU等待人工操作
\end{itemize}

由于以上两点,昂贵的机器资源在大部分时间内处于空闲状态,非常浪费。

优点:交互性好,即用户可以得到机器的立即响应。

\subsubsection{单道批处理系统}

\textbf{批处理系统}:在计算机上加载一个专门监控软件,在其控制下,计算机
能够自动地、成批地处理一个或多个用户的一批作业。

批处理系统的特点:
\begin{itemize}
  \item 系统吞吐量大
  \item 资源利用率高
  \item 平均周转时间短
  \item 无交互能力(缺点)
\end{itemize}

\textbf{单道批处理系统}:先把一批作业输入到磁带上,在监控程序的控制下使这批作业一个接一个地处理。
内存中始终只保持一道作业。

代表系统:IBM的FMS(Fortran Monitoring System),1960s

一定程度上改善了资源浪费,但失去了交互性。单个程序独占系统,程序I/O时CPU空等。

单道批处理系统的特点:自动性、顺序性、单道性

\subsubsection{多道批处理系统}

出现的动力是希望提高资源利用率和系统吞吐量。

代表系统(也是第一个):IBM的OS/360,1960s

由作业调度程序按照一定的算法,从后备队列中选择若干个作业调入内存,使它们共享CPU资源。

相比单道批处理系统,多道批处理系统的资源利用率更高,但仍然没有交互性。

特点:多道性、无序性、调度性、宏观上并行、微观上串行

\subsubsection{分时系统}

出现的动力是改善交互性。

分时系统:在一台主机上连接若干个终端,允许多个用户通过终端以交互方式共享主机资源。

代表系统:CTSS(1962)、Multics(1964)

特点:
\begin{itemize}
  \item 多路性:众多联机用户可以同时使用同一台计算机
  \item 独立性:各终端用户感觉到自己独占了计算机
  \item 及时性:用户的请求能在很短时间内得到响应
  \item 交互性:用户与计算机之间可进行“会话”
\end{itemize}

\subsubsection{实时系统}

实时计算:系统的正确性不仅由计算的逻辑结果来确定,而且还取决于产生结果的时间。

常见的实时系统:
\begin{itemize}
  \item 工业控制系统、武器控制系统
  \item 信息查询系统
  \item 多媒体系统
  \item 嵌入式系统
\end{itemize}

代表系统:WinCE,嵌入式Linux,ucOSII,VxWorks等

特点:实时性、高安全性、高可靠性(效率放第二位)、整体性强、会话能力要求不高

\subsubsection{网络操作系统和分布式计算机系统}

网络操作系统:将网络中的多台计算机连接起来。

分布式计算机系统:用于管理分布式计算机系统的操作系统。

\subsection{操作系统运行环境}

\subsubsection{处理器运行模式}

\begin{itemize}
  \item {\kaishu 特权指令},指不允许用户直接使用的指令。
  \item {\kaishu 非特权指令},指允许用户直接使用的指令。
\end{itemize}

CPU的运行模式:
\begin{itemize}
  \item {\kaishu 用户态}(目态)
  \item {\kaishu 内核态}(核心态、管态)
\end{itemize}

CPU在内核态可以执行特权指令,在用户态则只能执行非特权指令。应用程序运行在用户态,操作系统内核运行在内核态。

大多数操作系统内核包括:
\begin{itemize}
  \item 时钟管理
  \item 中断机制
  \item 原语
  \item 系统控制的数据结构及处理
\end{itemize}

\subsubsection{中断和异常的概念}

可直接参考组成原理相关内容。

\subsubsection{系统调用}



\section{进程与线程}

\section{内存管理}

\section{文件管理}

\section{输入输出(I/O)管理}

\end{document}