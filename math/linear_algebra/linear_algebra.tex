\PassOptionsToPackage{dvipsnames}{xcolor}

% draft会跳过文档中的所有图片。正式导出时需要删掉draft参数。
\documentclass[12pt, a4paper, oneside]{ctexart}

\usepackage{amsmath}
\usepackage{amssymb}
\usepackage{bm}
\usepackage{graphicx}
\usepackage{mathrsfs}
\usepackage{geometry}
\usepackage{framed}
\usepackage{caption}
\usepackage{listings}
\usepackage{fancyhdr}
\usepackage{booktabs}
\usepackage{makecell}
\usepackage{indentfirst}
\usepackage{authblk}
\usepackage{multicol}
% \usepackage{draftwatermark}       % 需要应用水印时取消注释
\usepackage{enumitem}
\usepackage[hidelinks]{hyperref}
\usepackage{tikz}
\usepackage{ulem}
\usetikzlibrary{positioning, shapes.geometric}

% 分栏线宽
\columnseprule=0.4pt

% 定制第二级无序列表的点样式
\setlist[itemize,2]{label=$\diamond$}

% 页边距
\geometry{a4paper, scale=0.8}

\pagestyle{fancy}

% 调整页眉高度,用于去除警告
\setlength{\headheight}{25pt}

\fancyhf{}      % 清空页眉页脚设置
\fancyhead[L] {
    % 工大计算机系logo
    \includegraphics[height=7mm]{./images/logo1.jpg}
}
\fancyhead[C]{《线性代数》复习}
\fancyhead[R]{\leftmark}    % 右侧页眉:当前章标题

% 页脚居中放置页码
\fancyfoot[C]{\thepage}

% 设置章节标题自动编号的格式
\ctexset{
  section/number=\chinese{section},
%   subsection/name={,},
%   subsection/number=\chinese{subsection}
}

% 行距。ctexart默认值为1.3
\linespread{1.5}

\lstset{
  language=pascal,
  basicstyle=\ttfamily,
  frame=single,
  numbers=left
}

% \SetWatermarkText{Eslzzyl整理}            % 设置水印内容
% \SetWatermarkLightness{0.9}             % 设置水印透明度 0-1
% \SetWatermarkScale{0.8}                   % 设置水印大小 0-1

\renewcommand{\headrulewidth}{1pt}  %页眉线宽,设为0可以去页眉线
\renewcommand{\footrulewidth}{1pt}  %脚注线的宽度

\definecolor{shadecolor}{RGB}{241, 241, 255}

\title{
    \includegraphics[width=0.3\textwidth]{images/hfut-badge.pdf}
    
    \vspace{20pt}
    《线性代数》总复习
}
\author{Eslzzyl}
\date{\today}

\newcounter{problemname}
\newenvironment{problem}{\begin{shaded}\stepcounter{problemname}\par\noindent\textbf{例题\arabic{problemname}. }}{\end{shaded}\par}
\newenvironment{solution}{\begin{shaded}\par\noindent\textbf{解答:}}{\end{shaded}\par}
% \newenvironment{solution}{\par\noindent\textbf{答案. }}{\par}
% \newenvironment{note}{\par\noindent\textbf{例题\arabic{problemname}的注记. }}{\\\par}
\newenvironment{note}{\par\noindent\textbf{注记. }}{\par}

\begin{document}

\maketitle
\newpage
\tableofcontents
\vspace{20pt}
% 如果在目录处有备注,可以写在这里。

\newpage

\section{行列式}

\subsection{行列式的性质}

行列式中的行与列具有同等的地位,行列式的性质凡是对行成立的对列也同样成立,反之亦然。

\begin{enumerate}
  \item 行列式和它的转置行列式相等。
  \item 对换行列式的两行(列),行列式变号。
  \begin{itemize}
    \item 推论:如果行列式有两行(列)完全相同,则此行列式等于零。(把这两行对换,得$D=-D$)
  \end{itemize}
  \item 行列式的某一行(列)中所有的元素都乘以同一个数$k$,等于用$k$乘以此行列式。
  \begin{itemize}
    \item 推论:行列式中某一行(列)的所有元素的公因子可以提到行列式的外面。
  \end{itemize}
  \item 行列式中如果有两行(列)元素成比例,则此行列式等于零。(由上面的两个推论得出)
  \item 若行列式的某一行(列)的元素都是两数之和,则行列式可以拆开。
  \item 把行列式的某行(列)乘以同一数然后加到另一行(列)对应的元素上去,行列式不变。
\end{enumerate}

\subsection{行列式按行(列)展开}

\begin{itemize}
  \item $M_{ij}$:余子式
  \item $A_{ij}$:代数余子式
\end{itemize}

\begin{enumerate}
  \item 行列式等于它的任一行(列)的各元素与其对应的代数余子式乘积之和。
  \item 行列式某一行(列)的元素与另一行(列)的对应元素的代数余子式乘积之和等于零。
\end{enumerate}

\subsection{一些特殊的行列式}

\subsubsection{范德蒙德(Vandermonde)行列式}

\begin{equation}
  D_n=\begin{vmatrix}
    1 & 1 & \cdots & 1 \\
    x_1 & x_2 & \cdots & x_n \\
    x_1^2 & x_2^2 & \cdots & x_n^2 \\
    \vdots & \vdots & & \vdots \\
    x_1^{n-1} & x_2^{n-1} & \cdots & x_n^{n-1}
  \end{vmatrix}=\prod_{n\geq i\geq j\geq 1} (x_i-y_i)
\end{equation}

\section{矩阵及其运算}

\subsection{线性方程组和矩阵}

\subsubsection{线性方程组}

\textbf{齐次线性方程组}:即常数项全为0的线性方程组:

\begin{equation}
  \left\{\begin{matrix}
    a_{11}x_1+a_{12}x_2+\cdots +a_{1n}x_n=0 \\
    a_{21}x_1+a_{22}x_2+\cdots +a_{2n}x_n=0 \\
    \cdots \\
    a_{m1}x_1+a_{m2}x_2+\cdots +a_{mn}x_n=0
   \end{matrix}\right.
\end{equation}

齐次线性方程组\textbf{一定}有零解($x_1=x_2=\cdots =x_n=0$),\textbf{可能}有非零解。

\textbf{非齐次线性方程组}:即常数项不全为0的线性方程组:

\begin{equation}
  \left\{\begin{matrix}
    a_{11}x_1+a_{12}x_2+\cdots +a_{1n}x_n=b_1 \\
    a_{21}x_1+a_{22}x_2+\cdots +a_{2n}x_n=b_2 \\
    \cdots \\
    a_{m1}x_1+a_{m2}x_2+\cdots +a_{mn}x_n=b_2
   \end{matrix}\right.
\end{equation}

\subsubsection{矩阵}

行数与列数都为$n$的矩阵称为$n$阶矩阵或$n$阶方阵。$n$阶矩阵$\mathbf{A}$也记作$\mathbf{A}_n$。

也就是说,只要见到“$n$阶矩阵”或者$\mathbf{A}_n$这样的说法时,就表示矩阵是方阵。

\begin{itemize}
  \item 行矩阵(行向量):

  \begin{equation*}
    \mathbf{A}=(a_1,a_2,\cdots,a_n)
  \end{equation*}

  \item {列矩阵(列向量)}
  
  \begin{equation*}
    \mathbf{B}=\begin{pmatrix}
      b_1 \\ 
      b_2 \\
      \cdots \\
      b_m 
     \end{pmatrix}
  \end{equation*}

\end{itemize}

行数列数都相等的矩阵称为\textbf{同型矩阵}。

零矩阵:$\mathbf{O}$

线性变换和矩阵之间存在一一对应关系。

除了主对角线之外的其他所有元素均为0的矩阵叫做\textbf{对角矩阵},记作\footnote{同济线代中对角矩阵的符号是字母A去掉一横}

\begin{equation*}
  \mathbf{A}=\text{diag}(\lambda_1, \lambda_2,\cdots,\lambda_n)
\end{equation*}

单位矩阵$\mathbf{E}$是特殊的对角矩阵。

\subsection{矩阵的运算}

\subsubsection{矩阵乘法}

若有两个矩阵$\mathbf{A}$、$\mathbf{B}$满足$\mathbf{AB}=\mathbf{O}$,不能就此得出$\mathbf{A}=\mathbf{O}$或$\mathbf{B}=\mathbf{O}$的结论。

\subsubsection{矩阵转置}

矩阵的转置:$(\mathbf{AB})^T=\mathbf{B}^T \mathbf{A}^T$

\subsubsection{对称矩阵}

若方阵$\mathbf{A}$满足$\mathbf{A}^T=\mathbf{A}$,则称$\mathbf{A}$为对称矩阵。显然所有的对角矩阵都是对称矩阵。

\subsubsection{矩阵的行列式}

方阵$\mathbf{A}$的行列式:$\text{det}\ \mathbf{A}$或$|\mathbf{A}|$。

有性质$|\mathbf{AB}|=|\mathbf{A}|\cdot |\mathbf{B}|$。因此,一般来说$\mathbf{AB}\neq\mathbf{BA}$,但总有$|\mathbf{AB}|=|\mathbf{BA}|$。

\subsubsection{伴随矩阵}

由行列式$|\mathbf{A}|$的各个元素的代数余子式$A_{ij}$所构成的矩阵
\begin{equation*}
  \mathbf{A}^{*}=\begin{pmatrix}
    A_{11} & A_{21} & \cdots & A_{n1} \\
    A_{12} & A_{22} & \cdots & A_{n2} \\
    \vdots & \vdots & & \vdots \\
    A_{1n} & A_{2n} & \cdots & A_{nn}
  \end{pmatrix}
\end{equation*}

称为$\mathbf{A}$的伴随矩阵。注意这里有一个隐含的转置运算。

$2\times 2$矩阵的伴随矩阵:主对角线元素交换位置,副对角线元素变号。

有如下规律:
\begin{equation*}
  \mathbf{AA}^{*}=\mathbf{A}^{*}\mathbf{A}=|\mathbf{A}|\mathbf{E}
\end{equation*}

\subsection{逆矩阵}

\subsubsection{定义、性质和求法}

定义:对于$n$阶矩阵$\mathbf{A}$,如果有一个$n$阶矩阵$B$,使
\begin{equation*}
  \mathbf{AB}=\mathbf{BA}=\mathbf{E}
\end{equation*}

则说矩阵$\mathbf{A}$是可逆的,称$\mathbf{B}$是$\mathbf{A}$的逆矩阵。记作$\mathbf{B}=\mathbf{A}^{-1}$。此时$\mathbf{B}$是唯一的。

单元素矩阵的逆矩阵仍然是一个单元素矩阵,这唯一的元素等于原矩阵中唯一元素的倒数。

性质:
\begin{enumerate}
  \item 若$\mathbf{A}$可逆,则$|\mathbf{A}|\neq 0$,此时$\mathbf{A}$称为非奇异矩阵(反之则为奇异矩阵)。
  \item 若$|\mathbf{A}|\neq 0$,则$\mathbf{A}$可逆,且
  \begin{equation*}
    \mathbf{A}^{-1}=\frac{1}{|\mathbf{A}|}\mathbf{A}^{*}。
  \end{equation*}
\end{enumerate}

运算规律:
\begin{enumerate}
  \item $(\mathbf{A}^{-1})^{-1}=\mathbf{A}$
  \item $(\lambda \mathbf{A})^{-1}=\frac{1}{\lambda}\mathbf{A}^{-1}$
  \item $\mathbf{A}$、$\mathbf{B}$为同阶矩阵且均可逆,则有$(\mathbf{AB})^{-1}=\mathbf{B}^{-1}\mathbf{A}^{-1}$。
  \item 若$\mathbf{A}$可逆,则$\mathbf{A}^T$亦可逆,且$(\mathbf{A}^T)^{-1}=(\mathbf{A}^{-1})^T$。
\end{enumerate}

\subsubsection{逆矩阵的初步应用}

设$\varphi (x)=a_0+a_1 x+\cdots+a_m x^m$为$x$的$m$次多项式,$\mathbf{A}$为$n$阶矩阵,记
\begin{equation*}
  \varphi(\mathbf{A})=a_0 \mathbf{E}+a_1 \mathbf{A}+\cdots +a_m \mathbf{A}^m
\end{equation*}

$\varphi(\mathbf{A})$称为矩阵$\mathbf{A}$的$m$次多项式。

\subsection{克拉默法则}

如果线性方程组
\begin{equation*}
  \left\{\begin{matrix}
    a_{11}x_1+a_{12}x_2+\cdots +a_{1n}x_n=b_1 \\
    a_{21}x_1+a_{22}x_2+\cdots +a_{2n}x_n=b_2 \\
    \cdots \\
    a_{m1}x_1+a_{m2}x_2+\cdots +a_{mn}x_n=b_2
   \end{matrix}\right.
\end{equation*}

的系数矩阵的行列式不等于零,即
\begin{equation*}
  |\mathbf{A}|=\begin{vmatrix}
    a_{11} & \cdots & a_{1n} \\
    \vdots & & \vdots \\
    a_{n1} & \cdots & a_{nn}
  \end{vmatrix}\neq 0
\end{equation*}

那么,这方程组有唯一解
\begin{equation*}
  x_1=\frac{|\mathbf{A}_1|}{|\mathbf{A}|},\quad x_2=\frac{|\mathbf{A}_2|}{|\mathbf{A}|},\cdots, \quad x_n=\frac{|\mathbf{A}_n|}{|\mathbf{A}|}
\end{equation*}

其中$\mathbf{A}_j$是把系数矩阵$\mathbf{A}$中第$j$列的元素用方程组右端的常数项代替后所得到的$n$阶矩阵。

\subsection{矩阵分块法}

\subsubsection{分块矩阵的相加和数乘}

分块矩阵的相加、数乘是显然的。其中相加要求两个矩阵的行数、列数都相同,且采用相同的分块方法。

\subsubsection{分块矩阵相乘}

分块矩阵相乘:设$\mathbf{A}$为$m\times l$矩阵,$\mathbf{B}$为$l\times n$矩阵,分块成
\begin{equation*}
  \mathbf{A}=\begin{pmatrix}
    \mathbf{A}_{11} & \cdots & \mathbf{A}_{1t} \\
    \vdots & & \vdots \\
    \mathbf{A}_{s1} & \cdots & \mathbf{A}_{st}
  \end{pmatrix},\quad
  \mathbf{B}=\begin{pmatrix}
    \mathbf{B}_{11} & \cdots & \mathbf{B}_{1r} \\
    \vdots & & \vdots \\
    \mathbf{B}_{t1} & \cdots & \mathbf{B}_{tr}
  \end{pmatrix}
\end{equation*}

其中$\mathbf{A}_{i1},\mathbf{A}_{i2},\cdots,\mathbf{A}_{it}$的列数分别等于$\mathbf{B}_{1j},\mathbf{B}_{2j},\cdots,\mathbf{B}_{tj}$的行数,那么
\begin{equation*}
  \mathbf{AB}=\begin{pmatrix}
    \mathbf{C}_{11} & \cdots & \mathbf{C}_{1r} \\
    \vdots & & \vdots \\
    \mathbf{C}_{s1} & \cdots & \mathbf{B}_{sr}
  \end{pmatrix}
\end{equation*}

其中
\begin{equation*}
  \mathbf{C}_{ij}=\sum_{k=1}^t \mathbf{A}_{ik} \mathbf{B}_{kj}\quad (i=1,\cdots,s;\ j=1,\cdots,r).
\end{equation*}

\subsubsection{分块矩阵的转置}

分块矩阵的转置:矩阵本身要转置,各个子块自身也要转置。设
\begin{equation*}
  \mathbf{A}=\begin{pmatrix}
    \mathbf{A}_{11} & \cdots & \mathbf{A}_{1r} \\
    \vdots & & \vdots \\
    \mathbf{A}_{s1} & \cdots & \mathbf{A}_{sr}
  \end{pmatrix}
\end{equation*}

则
\begin{equation*}
  \mathbf{A}^T=\begin{pmatrix}
    \mathbf{A}_{11}^T & \cdots & \mathbf{A}_{s1}^T \\
    \vdots & & \vdots \\
    \mathbf{A}_{1r}^T & \cdots & \mathbf{A}_{sr}^T
  \end{pmatrix}
\end{equation*}

\subsubsection{分块对角矩阵}

设$\mathbf{A}$为$n$阶方阵,若$\mathbf{A}$的分块矩阵只有在对角线上有非零子块,其余子块都是零矩阵,且在对角线上的子块都是方阵,即
\begin{equation*}
  \mathbf{A}=\begin{pmatrix}
    \mathbf{A}_1 & & & \mathbf{O} \\
    & \mathbf{A}_2 & & \\
    & & \ddots & \\
    \mathbf{O} & & & \mathbf{A}_s
  \end{pmatrix}
\end{equation*}

其中$\mathbf{A}_i$都是方阵,那么称$\mathbf{A}$为分块对角矩阵。

分块对角矩阵的行列式有如下性质:
\begin{equation*}
  |\mathbf{A}|=|\mathbf{A}_1|\ |\mathbf{A}_2|\ \cdots |\mathbf{A}_s|.
\end{equation*}

若$|\mathbf{A}_i|$均$\neq 0$,那么$|\mathbf{A}|$也$\neq 0$,且有
\begin{equation*}
  \mathbf{A}^{-1}=\begin{pmatrix}
    \mathbf{A}_1^{-1} & & & \mathbf{O} \\
    & \mathbf{A}_2^{-1} & & \\
    & & \ddots & \\
    \mathbf{O} & & & \mathbf{A}_s^{-1}
  \end{pmatrix}
\end{equation*}

行向量默认要带一个转置符号,因为向量默认是竖写的,而行向量是横的。列向量则不用带转置符号。

\section{矩阵的初等变换与线性方程组}

\subsection{矩阵的初等变换}

\subsubsection{定义}

矩阵的初等行变换:
\begin{itemize}
  \item 对换两行
  \item 以数$k\neq 0$乘某一行中的所有元
  \item 把某一行所有元的$k$倍加到另一行对应的元上去。
\end{itemize}

初等列变换可以简单类比。

若$\mathbf{A}$经过有限次初等行变换得到$\mathbf{B}$,就说$\mathbf{A}$与$\mathbf{B}$\textbf{行等价},记作$\mathbf{A}\overset{r}{\sim}\mathbf{B}$。同理\textbf{列等价}记作$\mathbf{A}\overset{c}{\sim}\mathbf{B}$。

矩阵的等价关系满足反身性、对称性和传递性。

非零矩阵若满足:
\begin{itemize}
  \item 非零行在零行的上面;
  \item 非零行的首非零元所在列在上一行(若有)的首非零元素所在列的右面
\end{itemize}

则称此矩阵为\textbf{行阶梯形矩阵}。若行阶梯形矩阵还满足:
\begin{itemize}
  \item 非零行的首非零元为1
  \item 首非零元所在列的其他元均为0
\end{itemize}

则称此矩阵为\textbf{行最简形矩阵}。任意矩阵总可以经过有限次初等\textbf{行}变换,变成行最简形矩阵。

对行最简形矩阵再进行初等\textbf{列}变换,使之满足条件:矩阵的左上角是一个单位矩阵,其余元素全为0。则此矩阵称为\textbf{标准形}。

\subsubsection{性质}

设$\mathbf{A}$和$\mathbf{B}$为$m\times n$矩阵,则:
\begin{itemize}
  \item $\mathbf{A}\overset{r}{\sim}\mathbf{B}$的充要条件是存在$m$阶可逆矩阵$\mathbf{P}$,使得$\mathbf{PA}=\mathbf{B}$。
  \item $\mathbf{A}\overset{c}{\sim}\mathbf{B}$的充要条件是存在$n$阶可逆矩阵$\mathbf{Q}$,使得$\mathbf{AQ}=\mathbf{B}$。
  \item $\mathbf{A}\sim\mathbf{B}$的充要条件是存在$m$阶可逆矩阵$\mathbf{P}$及$n$阶可逆矩阵$\mathbf{Q}$,使得$\mathbf{PAQ}=\mathbf{B}$。
\end{itemize}

由单位矩阵$\mathbf{E}$经过一次初等变换得到的矩阵称为\textbf{初等矩阵}。初等矩阵都是可逆的,且其逆矩阵是同一类型的初等矩阵。
\begin{itemize}
  \item 把单位矩阵的$i$、$j$两行对换,得到的初等矩阵记作$\mathbf{E}(i,j)$。用这矩阵左乘一个相应尺寸的矩阵$\mathbf{A}$,其效果就是将$\mathbf{A}$的$i$、$j$两行对换。
  \item 用数$k\neq 0$乘单位矩阵的第$i$行,得初等矩阵$\mathbf{E}(i(k))$。用这矩阵左乘$\mathbf{A}$,相当于用$k$乘$\mathbf{A}$的第$i$行。
  \item 用数$k$乘单位矩阵的第$j$行加到第$i$行上,得到的矩阵记作$\mathbf{E}(ij(k))$。用这矩阵左乘$\mathbf{A}$,相当于用$k$乘$\mathbf{A}$的第$j$行加到第$i$行上。
\end{itemize}

因此,对$\mathbf{A}$施行一次初等行变换,相当于在$\mathbf{A}$的左边乘相应的初等矩阵;对$\mathbf{A}$施行一次初等列变换,相当于在$\mathbf{A}$的右边乘相应的初等矩阵。

方阵$\mathbf{A}$可逆的充要条件是存在有限个初等矩阵$\mathbf{P}_1,\mathbf{P}_2,\cdots,\mathbf{P}_l$,使$\mathbf{A}=\mathbf{P}_1 \mathbf{P}_2 \cdots \mathbf{P}_l$。

推论:方阵$\mathbf{A}$可逆的充要条件是$\mathbf{A}\overset{r}{\sim}\mathbf{E}$。

一种求逆矩阵的方法:对矩阵$(\mathbf{A},\mathbf{E})$施行初等行变换,直到$\mathbf{A}$变成$\mathbf{E}$,此时原先的$\mathbf{E}$就变为$\mathbf{A}^{-1}$。$\mathbf{A}$能够通过初等行变换化成$\mathbf{E}$,就说明$\mathbf{A}$可逆(上面推论)。

\subsection{矩阵的秩}

在$m\times m$矩阵$\mathbf{A}$中,任取$k$行与$k$列($k\leq m,k\leq n$),位于这些行列交叉处的$k^2$个元素,不改变它们在$\mathbf{A}$中所处的位置次序而得到的$k$阶行列式,称为$\mathbf{A}$的$k$阶子式。

若$\mathbf{A}\overset{r}{\sim}\mathbf{B}$,则$\mathbf{A}$与$\mathbf{B}$中非零子式的最高阶数相等。

$\mathbf{A}$的最高阶非零子式的阶数称为$\mathbf{A}$的秩,记作$R(\mathbf{A})$。规定零矩阵的秩为0。

可逆矩阵的秩等于矩阵的阶数,因此可逆矩阵又称满秩矩阵,不可逆矩阵(奇异矩阵)又称降秩矩阵。

矩阵的秩的性质:
\begin{itemize}
  \item $0\leq R(\mathbf{A}_{m\times n})\leq\min\{m, n\}$
  \item $R(\mathbf{A}^T)=R(\mathbf{A})$
  \item 若$\mathbf{A}\sim\mathbf{B}$,则$R(\mathbf{A})=R(\mathbf{B})$。
  \begin{itemize}
    \item 因此,将矩阵化成行阶梯形矩阵,然后查非零行的数量,得到的就是矩阵的秩。
  \end{itemize}
  \item 若$\mathbf{P}$、$\mathbf{Q}$可逆,则$R(\mathbf{PAQ})=R(\mathbf{A})$。
  \item $\max\{R(\mathbf{A}),R(\mathbf{B})\}\leq R(\mathbf{A},\mathbf{B})\leq R(\mathbf{A})+R(\mathbf{B})$。
  \begin{itemize}
    \item 特别地,当$\mathbf{B}=\mathbf{b}$为非零列向量时,有$R(\mathbf{A})\leq R(\mathbf{A},\mathbf{b})\leq R(\mathbf{A})+1$。
  \end{itemize}
  \item $R(\mathbf{A}+\mathbf{B})\leq R(\mathbf{A})+R(\mathbf{B})$。
  \item $R(\mathbf{AB})\leq\min\{R(\mathbf{A}),R(\mathbf{B})\}$。
  \item 若$\mathbf{A}_{m\times n}\mathbf{B}_{n\times l}=\mathbf{O}$,则$R(\mathbf{A})+R(\mathbf{B})\leq n$。
\end{itemize}

\subsection{线性方程组的解}

如果线性方程组有解,就说它是相容的;如果无解就说它是不相容的。

定理:$n$元线性方程组$\mathbf{Ax}=\mathbf{b}$
\begin{itemize}
  \item 无解的充要条件的$R(\mathbf{A})<R(\mathbf{A},\mathbf{b})$;
  \item 有唯一解的充要条件是$R(\mathbf{A})=R(\mathbf{A},\mathbf{b})=n$;
  \item 有无限多解的充要条件是$R(\mathbf{A})=R(\mathbf{A},\mathbf{b})<n$;
  \item 有解的充要条件是$R(\mathbf{A})=R(\mathbf{A},\mathbf{b})$。
\end{itemize}

定理:$n$元齐次线性方程组$\mathbf{Ax}=\mathbf{0}$有非零解的充要条件是$R(\mathbf{A})<n$。

定理:设$\mathbf{AB}=\mathbf{C]}$,则$R(\mathbf{C})\leq \min\{R(\mathbf{A}),R(\mathbf{B})\}$。

\section{向量组的线性相关性}



\section{相似矩阵及二次型}

\section{线性空间与线性变换}

\end{document}