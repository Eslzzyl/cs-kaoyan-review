% draft会跳过文档中的所有图片。正式导出时需要删掉draft参数。
\documentclass[12pt, a4paper, oneside]{ctexart}

\usepackage{amsmath}
\usepackage{amssymb}
\usepackage{amsopn}
\usepackage{bm}
\usepackage{graphicx}
\usepackage{mathrsfs}
\usepackage{geometry}
\usepackage{framed}
\usepackage{xcolor}
\usepackage{caption}
\usepackage{listings}
\usepackage{fancyhdr}
\usepackage{booktabs}
\usepackage{makecell}
\usepackage{indentfirst}
\usepackage{authblk}
\usepackage{multicol}
% \usepackage{draftwatermark}       % 需要应用水印时取消注释
\usepackage{enumitem}
\usepackage[hidelinks]{hyperref}
\usepackage{tikz}
\usepackage{ulem}
\usetikzlibrary{positioning, shapes.geometric}

% 分栏线宽
\columnseprule=0.4pt

% 定制第二级无序列表的点样式
\setlist[itemize,2]{label=$\diamond$}

% 页边距
\geometry{a4paper, scale=0.8}

\pagestyle{fancy}

% 调整页眉高度,用于去除警告
\setlength{\headheight}{25pt}

\fancyhf{}      % 清空页眉页脚设置
\fancyhead[L] {
    % 工大计算机系logo
    \includegraphics[height=7mm]{../../../share/images/logo1.jpg}
}
\fancyhead[C]{精讲精练-习概部分}
\fancyhead[R]{\leftmark}    % 右侧页眉:当前章标题

% 页脚居中放置页码
\fancyfoot[C]{\thepage}

% 设置章节标题自动编号的格式
\ctexset{
  section/number=\chinese{section},
%   subsection/name={,},
%   subsection/number=\chinese{subsection}
}

% 行距。ctexart默认值为1.3
\linespread{1.2}

\lstset{
  language=c,
  basicstyle=\ttfamily,
  frame=single,
  numbers=left
}

% \SetWatermarkText{Eslzzyl整理}            % 设置水印内容
% \SetWatermarkLightness{0.9}             % 设置水印透明度 0-1
% \SetWatermarkScale{0.8}                   % 设置水印大小 0-1

\renewcommand{\headrulewidth}{1pt}  %页眉线宽,设为0可以去页眉线
\renewcommand{\footrulewidth}{1pt}  %脚注线的宽度

\definecolor{shadecolor}{RGB}{241, 241, 255}

\title{
    \includegraphics[width=0.3\textwidth]{../../../share/images/hfut-badge.pdf}
    
    \vspace{20pt}
    肖秀荣《精讲精练》\\ 之 \\ 习近平新时代中国特色社会主义思想概论
}
\author{Eslzzyl}
\date{\today}

\newcounter{problemname}
\newenvironment{problem}{\begin{shaded}\stepcounter{problemname}\par\noindent\textbf{例题\arabic{problemname}. }}{\end{shaded}\par}
\newenvironment{solution}{\begin{shaded}\par\noindent\textbf{解答:}}{\end{shaded}\par}
% \newenvironment{solution}{\par\noindent\textbf{答案. }}{\par}
% \newenvironment{note}{\par\noindent\textbf{例题\arabic{problemname}的注记. }}{\\\par}
\newenvironment{note}{\par\noindent\textbf{注记. }}{\par}

\begin{document}

\maketitle
\newpage
\tableofcontents
\vspace{20pt}
% 如果在目录处有备注,可以写在这里。

\textcolor{red}{习概时政性强,帽子提法很多。一般认为“五位一体”总体布局、“四个全面”战略布局是习概部分的重点。}

\newpage

\section{习思想及其历史地位}

\subsection{创立的社会历史条件}

\subsubsection{中国特色社会主义进入新时代}

对社会主要矛盾的科学判断,是制定党的路线方针政策的\textbf{基本依据}。

\vspace*{10pt}

两个没有变:
\begin{itemize}
  \item 我国仍处于并将长期处于社会主义初级阶段的基本国情没有变
  \item 我国是世界最大发展中国家的国际地位没有变
\end{itemize}

\subsection{习思想的科学体系}

\subsubsection{习思想的核心要义}

\textbf{坚持和发展中国特色社会主义},是改革开放依赖我们党全部理论和实践的鲜明主题,也是习思想的\textbf{核心要义}。

\subsubsection{习思想的主要内容}

即”十个明确“、”十四个坚持“、十三个方面成就

内容太多,只选一些帽子提法列于此处:

坚持党的领导、人民当家作主、依法治国有机统一是社会主义政治发展的\textbf{必然要求}。

全面依法治国是中特社会主义的\textbf{本质要求和重要保障}。

增进民生福祉是发展的\textbf{根本目的}。

用于自我革命,从严管党治党,是我们党\textbf{最鲜明的品格}。

\subsubsection{习思想的世界观和方法论}

必须坚持人民至上、自信自立、守正创新、问题导向、系统观念、胸怀天下。

人民性是马克思主义的本质属性。

\subsection{习思想的历史地位}

习思想是全党全国各族人民团结奋斗的共同思想基础,是全面建设社会主义现代化国家、实现中华民族伟大复兴中国梦的行动指南。

\section{坚持和发展中国特色社会主义的总任务}

\subsection{实现中华民族伟大复兴的中国梦}

\subsubsection{中华民族近代以来最伟大的梦想}

坚持和发展中国特色社会主义的总任务,是实现社会主义现代化和中华民族伟大复兴,在全面建成小康社会的基础上,分两步走在本世纪中叶建成富强民主文明和谐美丽的社会主义现代化强国。

中国梦是中华民族伟大复兴的形象表达。

\subsubsection{中国梦的科学内涵}

中国梦的\textbf{本质}是国家富强、民族振兴、人民幸福。

\begin{itemize}
  \item 国家富强、民族振兴是人民幸福的\textbf{基础和保障}
  \item 人民幸福是国家富强、民族振兴的\textbf{题中之义和必然要求},是国家富强、民族振兴的\textbf{根本出发点和落脚点}。
\end{itemize}

\subsection{建成社会主义现代化强国的战略安排}

略

\subsection{建设社会主义现代化强国的战略导向}

\textcolor{red}{这一节是帽子重灾区}

\vspace*{10pt}

\subsubsection{立足新发展阶段}

新发展阶段是指全面建成小康社会之后,开启全面建设社会主义现代化强国新征程的阶段。

\subsubsection{贯彻新发展理念}

新发展理念是十八大以来对经济社会发展提出的理论中最重要、最主要的。

新发展理念的科学内涵是创新协调绿色开放共享。

\begin{itemize}
  \item 创新是引领发展的第一动力。必须把创新摆在国家发展全局的核心位置,让创新贯穿党和国家一切工作。
  \begin{itemize}
    \item 将创新作为引领发展的第一动力,是引领经济发展新常态的根本之策。
  \end{itemize}
  \item 协调是持续健康发展的内在要求,解决的是不平衡问题。
  \item 绿色是永续发展的必要条件和人民对美好生活追求的重要体现。
  \item 开放是国家繁荣发展的必由之路。
  \item 共享是中国特色社会主义的本质要求,解决的是社会公平正义问题。
  \begin{itemize}
    \item 全民共享、全面共享、共建共享、渐进共享
  \end{itemize}
\end{itemize}

\subsubsection{构建新发展格局}

新发展格局,内涵是以国内大循环为主体、国内国际双循环相互促进。新发展格局是我国经济现代化的路径选择。

\begin{itemize}
  \item 构建新发展格局,关键在于经济循环的畅通无阻。必须坚持深化供给侧结构性改革这条\textbf{主线}。
  \item 构建新发展格局,最本质特征是实现高水平自立自强。
  \item 要坚持扩大内需这个战略基点
  \item 实行高水平对外开放
\end{itemize}

\section{“五位一体”总体布局}

\textbf{习概部分重点章。2019年开始每年都出单选和多选}

\subsection{推动高质量发展}



\end{document}