% draft会跳过文档中的所有图片。正式导出时需要删掉draft参数。
\documentclass[12pt, a4paper, oneside]{ctexart}

\usepackage{amsmath}
\usepackage{amssymb}
\usepackage{amsopn}
\usepackage{bm}
\usepackage{graphicx}
\usepackage{mathrsfs}
\usepackage{geometry}
\usepackage{framed}
\usepackage{xcolor}
\usepackage{caption}
\usepackage{listings}
\usepackage{fancyhdr}
\usepackage{booktabs}
\usepackage{makecell}
\usepackage{indentfirst}
\usepackage{authblk}
\usepackage{multicol, multirow}
% \usepackage{draftwatermark}       % 需要应用水印时取消注释
\usepackage{enumitem}
\usepackage[hidelinks]{hyperref}
\usepackage{tikz}
\usepackage{ulem}
\usetikzlibrary{positioning, shapes.geometric}

% 分栏线宽
\columnseprule=0.4pt

% 定制第二级无序列表的点样式
\setlist[itemize,2]{label=$\diamond$}

% 页边距
\geometry{a4paper, scale=0.8}

\pagestyle{fancy}

% 调整页眉高度,用于去除警告
\setlength{\headheight}{25pt}

\fancyhf{}      % 清空页眉页脚设置
\fancyhead[L] {
    % 工大计算机系logo
    \includegraphics[height=7mm]{../../../share/images/logo1.jpg}
}
\fancyhead[C]{精讲精练-习概部分}
\fancyhead[R]{\leftmark}    % 右侧页眉:当前章标题

% 页脚居中放置页码
\fancyfoot[C]{\thepage}

% 设置章节标题自动编号的格式
\ctexset{
  section/number=\chinese{section},
%   subsection/name={,},
%   subsection/number=\chinese{subsection}
}

% 行距。ctexart默认值为1.3
\linespread{1.5}

\lstset{
  language=c,
  basicstyle=\ttfamily,
  frame=single,
  numbers=left
}

% \SetWatermarkText{Eslzzyl整理}            % 设置水印内容
% \SetWatermarkLightness{0.9}             % 设置水印透明度 0-1
% \SetWatermarkScale{0.8}                   % 设置水印大小 0-1

\renewcommand{\headrulewidth}{1pt}  %页眉线宽,设为0可以去页眉线
\renewcommand{\footrulewidth}{1pt}  %脚注线的宽度

\definecolor{shadecolor}{RGB}{241, 241, 255}

\title{
    \includegraphics[width=0.3\textwidth]{../../../share/images/hfut-badge.pdf}
    
    \vspace{20pt}
    肖秀荣《精讲精练》\\ 之 \\ 习近平新时代中国特色社会主义思想概论
}
\author{Eslzzyl}
\date{\today}

\newcounter{problemname}
\newenvironment{problem}{\begin{shaded}\stepcounter{problemname}\par\noindent\textbf{例题\arabic{problemname}. }}{\end{shaded}\par}
\newenvironment{solution}{\begin{shaded}\par\noindent\textbf{解答:}}{\end{shaded}\par}
% \newenvironment{solution}{\par\noindent\textbf{答案. }}{\par}
% \newenvironment{note}{\par\noindent\textbf{例题\arabic{problemname}的注记. }}{\\\par}
\newenvironment{note}{\par\noindent\textbf{注记. }}{\par}

\begin{document}

\maketitle
\newpage
\tableofcontents
\vspace{20pt}
% 如果在目录处有备注,可以写在这里。

\textcolor{red}{习概时政性强,帽子提法很多。一般认为“五位一体”总体布局、“四个全面”战略布局是习概部分的重点。}

\newpage

\section{习思想及其历史地位}

\subsection{创立的社会历史条件}

\subsubsection{中国特色社会主义进入新时代}

对社会主要矛盾的科学判断,是制定党的路线方针政策的\textbf{基本依据}。

\vspace*{10pt}

两个没有变:
\begin{itemize}
  \item 我国仍处于并将长期处于社会主义初级阶段的基本国情没有变
  \item 我国是世界最大发展中国家的国际地位没有变
\end{itemize}

\subsection{习思想的科学体系}

\subsubsection{习思想的核心要义}

\textbf{坚持和发展中国特色社会主义},是改革开放依赖我们党全部理论和实践的鲜明主题,也是习思想的\textbf{核心要义}。

\subsubsection{习思想的主要内容}

即“十个明确”、“十四个坚持”、十三个方面成就

内容太多,只选一些帽子提法列于此处:

坚持党的领导、人民当家作主、依法治国有机统一是社会主义政治发展的\textbf{必然要求}。

全面依法治国是中特社会主义的\textbf{本质要求和重要保障}。

增进民生福祉是发展的\textbf{根本目的}。

勇于自我革命,从严管党治党,是我们党\textbf{最鲜明的品格}。

\subsubsection{习思想的世界观和方法论}

必须坚持人民至上、自信自立、守正创新、问题导向、系统观念、胸怀天下。

人民性是马克思主义的本质属性。

\subsection{习思想的历史地位}

习思想是全党全国各族人民团结奋斗的共同思想基础,是全面建设社会主义现代化国家、实现中华民族伟大复兴中国梦的行动指南。

\section{坚持和发展中国特色社会主义的总任务}

\subsection{实现中华民族伟大复兴的中国梦}

\subsubsection{中华民族近代以来最伟大的梦想}

坚持和发展中国特色社会主义的总任务,是实现社会主义现代化和中华民族伟大复兴,在全面建成小康社会的基础上,分两步走在本世纪中叶建成富强民主文明和谐美丽的社会主义现代化强国。

中国梦是中华民族伟大复兴的形象表达。

\subsubsection{中国梦的科学内涵}

中国梦的\textbf{本质}是国家富强、民族振兴、人民幸福。

\begin{itemize}
  \item 国家富强、民族振兴是人民幸福的\textbf{基础和保障}
  \item 人民幸福是国家富强、民族振兴的\textbf{题中之义和必然要求},是国家富强、民族振兴的\textbf{根本出发点和落脚点}。
\end{itemize}

\subsection{建成社会主义现代化强国的战略安排}

略

\subsection{建设社会主义现代化强国的战略导向}

\textcolor{red}{这一节是帽子重灾区}

\vspace*{10pt}

\subsubsection{立足新发展阶段}

新发展阶段是指全面建成小康社会之后,开启全面建设社会主义现代化强国新征程的阶段。($\neq$十八大后中国特色社会主义进入新时代)

\subsubsection{贯彻新发展理念}

新发展理念是十八大以来对经济社会发展提出的理论中最重要、最主要的。

新发展理念的\textbf{科学内涵是}创新协调绿色开放共享。

\begin{itemize}
  \item 创新是引领发展的第一动力。必须把创新摆在\textbf{国家发展全局的核心位置},让创新贯穿党和国家一切工作。
  \begin{itemize}
    \item 将创新作为引领发展的第一动力,是引领经济发展新常态的\textbf{根本之策}。
  \end{itemize}
  \item 协调是持续健康发展的内在要求,解决的是不平衡问题。
  \item 绿色是永续发展的必要条件和人民对美好生活追求的重要体现。
  \item 开放是国家繁荣发展的\textbf{必由之路}。
  \item 共享是中国特色社会主义的\textbf{本质要求},解决的是\textbf{社会公平正义问题}。
  \begin{itemize}
    \item 全民共享、全面共享、共建共享、渐进共享
  \end{itemize}
\end{itemize}

\subsubsection{构建新发展格局}

新发展格局,内涵是以国内大循环为主体、国内国际双循环相互促进。新发展格局是我国经济现代化的路径选择。

\begin{itemize}
  \item 构建新发展格局,关键在于经济循环的畅通无阻。必须坚持深化供给侧结构性改革这条\textbf{主线}。
  \item 构建新发展格局,\textbf{最本质特征}是实现高水平自立自强。
  \item 要坚持扩大内需这个战略基点
  \item 实行高水平对外开放
\end{itemize}

\section{“五位一体”总体布局}

\textcolor{red}{习概部分重点章。2019年开始每年都出单选和多选}

\subsection{推动高质量发展(经济)}

\subsubsection{习近平经济思想}

主要内容:13条

\begin{enumerate}
  \item 加强党对经济工作的全面领导(我国经济发展的\textbf{根本保证})
  \item 坚持以人民为中心的发展思想(\textbf{根本立场})
  \item 进入新发展阶段(\textbf{历史方位})
  \item 坚持新发展理念(\textbf{指导原则})
  \item 构建新发展格局(\textbf{路径选择})
  \item 推动高质量发展(\textbf{鲜明主题})
  \item 坚持和完善社会主义基本经济制度(\textbf{制度基础})
  \item 部署实施国家重大发展战略(\textbf{战略举措})
  \item 坚持创新驱动发展(\textbf{第一动力})
  \item 大力发展制造业和实体经济(\textbf{主要着力点})
  \item 坚定不移全面扩大开放(\textbf{重要法宝})
  \item 统筹发展和安全(\textbf{重要保障})
  \item 坚持正确工作策略和方法(做好经济工作的\textbf{方法论})
\end{enumerate}

习近平经济思想是马克思主义政治经济学在当代中国、21世纪世界的最新理论成果,是新时代做好经济工作的\textbf{根本遵循}和\textbf{行动指南},也是我国经济高质量发展、全面建设社会主义现代化国家的\textbf{科学指南}。

\subsubsection{以推动高质量发展为主题}

高质量发展是全面建设社会主义现代化国家的\textbf{首要任务}。

发展是党执政兴国的第一要务。

坚持把发展经济的着力点放在\textbf{实体经济}上。

全面建设社会主义现代化国家,最艰巨最繁重的任务仍然在农村。

全国统一大市场的内涵是\textbf{高效规范、公平竞争、充分开放}。

新型工农城乡关系:工农互促、城乡互补、协调发展、共同繁荣

区域重大战略:
\begin{itemize}
  \item 京津冀协同发展
  \item 长三角一体化发展
  \item 长江经济带发展(共抓大保护、不搞大开发)
  \item 粤港澳大湾区发展
\end{itemize}

\subsubsection{实施科教兴国战略,强化现代化建设人才支撑}

教育、科技、人才是全面建设社会主义现代化国家的基础性、战略性支撑。

必须坚持:
\begin{itemize}
  \item 科技是第一生产力
  \item 人才是第一资源
  \item 创新是第一动力
\end{itemize}

培养什么人、怎样培养人、为谁培养人是教育的根本问题。

坚持创新在我国现代化建设全局中的\textbf{核心地位}。

坚持面向世界科技前沿、面向经济主战场、面向国家重大需求、面向人民生命健康,加快实现高水平科技自立自强。

\subsection{发展全过程人民民主,保障人民当家作主(政治)}

人民民主是社会主义的生命,是全面建设社会主义现代化国家的应有之义。

全过程人民民主是社会主义民主政治的本质属性,实现了\textbf{过程民主和成果民主、程序民主和实质民主、直接民主和间接民主、人民民主和国家意志相统一},是全链条、全方位、全覆盖的民主,是最广泛、最真实、最管用的民主。

民主选举、民主协商、民主决策、民主管理、民主监督

\subsubsection{坚持走中国特色社会主义政治发展道路}

走中……道路(标题),必须坚持党的领导、人民当家作主、依法治国有机统一。
\begin{itemize}
  \item 党的领导是人民当家作主和依法治国的\textbf{根本保证}
  \item 人民当家作主是社会主义民主政治的\textbf{本质特征(本质和核心)}
  \item 依法治国是党领导人民治理国家的\textbf{基本方式}
\end{itemize}
三者统一于我国社会主义民主政治的伟大实践。

中国社会主义民主的两种重要形式:
\begin{itemize}
  \item 人民通过选举、投票形式权利
  \item 人民内部各方面在重大决策之前进行充分协商、尽可能就共同性问题取得一致意见
\end{itemize}

\subsubsection{加强人民当家作主制度保障}

\begin{table}
  \centering
  \caption{人民当家作主的制度安排}
  \begin{tabular}{|c|c|}
    \hline
    国体(根本政治性质) & 人民民主专政 \\ \hline
    政治、根本政治制度 & 人民代表大会制度 \\ \hline
    \multirow{3}{*}{基本政治制度} & 中国共产党领导的多党合作和政治协商制度 \\ \cline{2-2}
    & 民族区域自治制度 \\ \cline{2-2}
    & 基层群众自治制度 \\ \hline
    社会主义协商民主 & \makecell{和选举民主相对应,是中国社会主义\\ 民主政治的特有形式和独特优势} \\
    \hline
  \end{tabular}
\end{table}

\begin{enumerate}
  \item {\bf 人民代表大会制度}
  
  人民代表大会制度是支撑我国治理体系和治理能力的\textbf{根本政治制度},是实现我国全过程人民民主的重要制度载体。

  人民代表大会制度的重要作用(三个有效保证):
  \begin{itemize}
    \item 有效保证国家沿着社会主义道路前进
    \item 有效保证国家治理跳出治乱兴衰的历史周期率
    \item 有效保证国家政治生活既充满活力又安定有序
  \end{itemize}

  \item {\bf 中国共产党领导的多党合作和政治协商制度}
  
  坚持中国共产党领导,始终同中国共产党同心同德、团结奋斗,是多党合作的\textbf{根本政治基础}。

  人民政协的主题是\textbf{团结和民主}。

  人民政协的主要职能是\textbf{政治协商、民主监督、参政议政}。

  把坚持和发展中国特色社会主义作为巩固共同思想政治基础的主轴。

  把服务“两个一百年”奋斗目标作为\textbf{工作主线}。

  把加强思想政治引领、广泛凝聚共识作为\textbf{中心环节}。

  \item {\bf 民族区域自治制度}
  
  民族区域自治制度是党解决民族问题的基本政策。

  民族区域自治制度的核心是保障少数民族当家作主,管理本民族、本地方事务的权利。

  \item {\bf 基层群众自治制度}
  
  基层民主是全过程人民民主的重要体现。

  目前,中国已经建立了以农村村民委员会、城市居民委员会和企业职工代表大会为主要内容的基层民主自治体系。

  \item {\bf 社会主义协商民主是适合中国国情、有效管用的民主形式}
  
  有事好商量,众人的事情由众人商量,是\textbf{人民民主的真谛}。
\end{enumerate}

\subsubsection{巩固和发展爱国统一战线}

人心向背、力量对比是决定党和人民事业成败的关键,是\textbf{最大的政治}。

统战工作的\textbf{本质要求}是大团结大联合。

今天的爱国统一战线是由中国共产党领导的,有各民主党派和各人民团体参加的,包括全体社会主义劳动者、社会主义事业的建设者、拥护社会主义的爱国者、拥护祖国统一和致力于中华民族伟大复兴的爱国者的联盟。

文化认同是最深层次的认同,是民族团结之根、民族和睦之魂。

\subsection{建设社会主义文化强国(文化)}

\subsubsection{坚持马克思主义在意识形态领域指导地位的根本制度}

\subsubsection{培育和践行社会主义核心价值观}

社会主义核心价值体系的四个组成方面:
\begin{itemize}
  \item 马克思主义指导思想
  \item 中国特色社会主义共同理想
  \item 以爱国主义为核心的民族精神和以改革创新为核心的时代精神
  \item 社会主义荣辱观
\end{itemize}

社会主义核心价值观是社会主义核心价值体系的内核凝练和集中表达,体现着社会主义核心价值体系的根本性质和基本特征。

\subsubsection{坚定文化自信,繁荣发展社会主义文化}

讲好中国故事是树立当代中国良好形象、提升国家软实力的重要战略任务,是提高中华文化影响力的基本途径。

文化产业的意识形态属性是本质属性。

\subsection{加强以民生为重点的社会建设(社会)}

\subsubsection{提高保障和改善民生水平的重要性}

为民造福是立党为公、执政为民的\textbf{本质要求}。

民生是人民幸福之基、社会和谐之本。

\subsubsection{在发展中保障和改善民生}

就业是最基本的民生。

\subsection{建设美丽中国(生态)}

\subsubsection{坚持习近平生态文明思想}

生态文明建设的根本保证:党对生态文明建设的全面领导。

生态文明建设的核心理念:绿水青山就是金山银山。

生态文明建设的宗旨要求:良好生态环境是最普惠的民生福祉

生态文明建设的战略路径:绿色发展是发展观的深刻革命

生态文明建设的系统观念:统筹山水林田湖草沙系统治理

生态文明建设的制度保障:用最严格制度、最严密法治保护生态环境

生态文明建设的社会力量:把建设美丽中国转化为全体人民的自觉行动

生态文明建设的全球倡议:共谋全球生态文明建设之路

\subsubsection{推动绿色发展,促进人与自然和谐共生}

尊重自然、顺应自然、保护自然,是全面建设社会主义现代化国家的内在要求。

推动经济社会发展绿色化、低碳化是实现高质量发展的关键环节。

\section{“四个全面”战略布局}

\textcolor{red}{习概重点章}

属于我国\textbf{根本}制度的有:
\begin{itemize}
  \item 党的领导根本制度
  \item 马克思主义在意识形态领域的指导地位的根本制度
  \item 人民代表大会根本制度
  \item 党对人民军队的绝对领导根本制度
\end{itemize}

属于我国\textbf{基本}制度的有:
\begin{itemize}
  \item 中国共产党领导的多党合作和政治协商基本政治制度
  \item 民族区域自治基本政治制度
  \item 基层群众自治基本政治制度
  \item 社会主义基本经济制度
\end{itemize}

\subsection{全面建设社会主义现代化国家}

\subsubsection{全面建设社会主义现代化国家的新征程的开启}

全面建设社会主义现代化国家是战略目标,在“四个全面”战略布局中,居于引领地位。
\begin{itemize}
  \item 全面深化改革是全面建设社会主义现代化国家的\textbf{动力源泉}。
  \item 全面依法治国是全面深化改革的\textbf{法制保障}和全面建设社会主义现代化国家的\textbf{重要基石}。
  \item 全面从严治党则是【其他三个】的\textbf{必然要求}和\textbf{根本保证}。
\end{itemize}

\subsubsection{中国式现代化的中国特色和本质要求}

中国式现代化理论是科学社会主义的最新重大成果。

中国式现代化的中国特色:
\begin{itemize}
  \item 是人口规模巨大的现代化(\textbf{显著特征})
  \item 是全体人民共同富裕的现代化(\textbf{本质特征})
  \item 是物质文明和精神文明相协调的现代化(崇高追求)
  \begin{itemize}
    \item 物质富足、精神富有是社会主义现代化的\textbf{根本要求}。
  \end{itemize}
  \item 是人与自然和谐共生的现代化(鲜明特点)
  \item 是走和平发展道路的现代化(突出特征)
\end{itemize}

\subsubsection{以中国式现代化全面推进中华民族伟大复兴必须牢牢把握的重大原则}

太多了,只记一些帽子

坚持中国共产党领导,是中国式现代化最鲜明的特征和最突出的优势,是推进中国式现代化必须坚持的最高原则。

\subsection{全面深化改革}

\subsubsection{改革开放}

\subsubsection{坚定不移推进全面深化改革}

\subsubsection{坚持和完善中国特色社会主义制度,推进国家治理现代化}

全面深化改革的总目标是坚持和完善中国特色社会主义制度(根本方向)、推进国家治理体系和治理能力现代化(实现路径)。

我国国家治理体系和治理能力是中国特色社会主义制度及其执行能力的集中体现。

党的领导制度是我国的\textbf{根本领导制度}。

\subsection{全面依法治国}

\subsubsection{坚持习近平法治思想}

全面依法治国是中国特色社会主义的\textbf{本质要求和重要保障}。

习近平法治思想的主要内容是“十一个坚持”:
\begin{enumerate}
  \item 坚持党对全面依法治国的领导
  \item 坚持以人民位中心
  \item 坚持中国特色社会主义法治道路
  \item 坚持依宪治国、依宪执政
  \begin{itemize}
    \item 全面贯彻实施宪法是全面依法治国的首要任务
  \end{itemize}
  \item 坚持在法治轨道上推进国家治理体系和治理能力现代化
  \item 坚持建设中国特色社会主义法治体系
  \begin{itemize}
    \item 中国特色社会主义法治体系是推进全面依法治国的总抓手
  \end{itemize}
  \item 坚持依法治国、依法执政、依法行政共同推进,法治国家、法治政府、法治社会一体建设
  \item 坚持全面推进科学立法、严格执法、公正司法、全民守法
  \item 坚持统筹推进国内法治和涉外法治
  \item 坚持建设德才兼备的高素质法治工作队伍
  \item 坚持抓住领导干部这个“关键少数”
\end{enumerate}

\subsubsection{走中国特色社会主义法治道路}

中国特色社会主义法治道路的核心要义是坚持党的领导,坚持中国特色社会主义制度,贯彻中国特色社会主义法治理论。

\subsubsection{深化依法治国实践}

\begin{itemize}
  \item 法治国家是法治建设的目标
  \item 法治政府是建设法治国家的重点
  \item 法治社会是构筑法治国家的基础
\end{itemize}

\subsection{全面从严治党}

\subsubsection{全面从严治党是伟大的自我革命}

全面从严治党是党永葆生机活力、走好新的赶考之路的必由之路。

勇于自我革命,是我们党最鲜明的品格、我们党最大的优势,是……第二个历史答案。

全面从严治党是新时代党的自我革命的伟大实践,开辟了百年大党自我革命的新境界。

全面从严治党,核心是加强党的领导,基础在全面,关键在严,要害在治。

\subsubsection{新时代党的建设总要求}

新时代党的建设的目的:坚持和加强党的全面领导。这也是新时代党的建设的原则。

新时代党的建设的方针:坚持党要管党、全面从严治党。

\subsubsection{把全面从严治党引向深入}

党的政治建设是党的\textbf{根本性建设},决定党的建设方向和效果。政治属性是政党第一位的属性,政治建设是政党建设的内在要求。

思想建设是党的\textbf{基础性建设}。要把坚定理想信念作为党的思想建设的首要任务。

人民立场是党的根本政治立场。

加强纪律建设是全面从严治党的治本之策。

\section{实现中华民族伟大复兴的重要保障}

\subsection{坚持总体国家安全观}

国家安全是民族复兴的根基,社会稳定是国家强盛的前提。

\subsubsection{总体国家安全观的提出及其依据}

国家安全是安邦定国的重要基石,是人民幸福安康的基本要求,维护国家安全是全国各族人民根本利益所在。

总体国家安全观,坚持国家利益至上,以人民安全为\textbf{宗旨}、以政治安全为\textbf{根本}、以经济安全为\textbf{基础}、以军事科技文化社会安全为\textbf{保障}、以促进国际安全为\textbf{依托}。

\subsubsection{推进国家安全体系和能力现代化,坚决维护国家安全和社会稳定}

在社会基层坚持和发展新时代“枫桥经验”。

\subsection{加快国防和军队现代化}

\subsubsection{坚持习近平强军思想}

\subsubsection{实现党在新时代的强军目标}

党在新时代的强军目标是建设一支听党指挥、能打胜仗、作风优良的人民军队,把人民军队建设成为世界一流军队。

党对军队绝对领导的根本原则和制度,\textbf{发端于南昌起义,奠基于三湾改编,定型于古田会议},是人民军队完全区别于一切旧军队的政治特质和根本优势。

能打仗、打胜仗是党和人民对军队的根本要求。

牢固树立战斗力这个唯一的根本的标准。

\subsubsection{巩固提高一体化国家战略体系和能力}

经济建设是国防建设的基本依托。国防建设是我国现代化建设的战略任务。

\subsection{坚持“一国两制”,推进祖国统一}

\subsubsection{全面准确贯彻“一国两制”方针}

维护国家主权、安全、发展利益是“一国两制”方针的最高原则。

\subsubsection{实现祖国完全同意}

两岸长期存在的政治分歧问题是影响两岸关系行稳致远的总根子。“台独”分裂势力及其活动是两岸和平发展的最大障碍,是台海和平稳定的最大威胁。

\section{中国特色大国外交}

\subsection{坚持习近平外交思想}

\subsubsection{习近平外交思想的核心要义}

共10条,略

\subsubsection{新时代对外工作的根本遵循}

\subsection{坚持走和平发展道路}

\subsubsection{坚持独立自主和平外交政策}

\subsubsection{推动建设新型国际关系}

\subsection{推动构建人类命运共同体}

\subsubsection{人类命运共同体的内涵}

构建人类命运共同体,核心就是“建设持久和平、普遍安全、共同繁荣、开放包容、清洁美丽的世界”。持久和平是基石,普遍安全是保障,共同繁荣是核心,开放包容是特征,清洁美丽是底色。

\subsubsection{促进“一带一路”国际合作}

党的十九大提出要以“一带一路”建设为重点,坚持引进来和走出去并重,遵循共商共建共享原则,加强创新能力开放合作,形成陆海内外联动、东西双向互济的开放格局。

\section{坚持和加强党的领导}

\subsection{实现中华民族伟大复兴关键在党}

\subsubsection{中国共产党的领导地位是历史和人民的选择}

\subsubsection{中国特色社会主义最本质的特征}

党的领导是中国特色社会主义最本质的特征。

党的领导是实现社会主义现代化和民族复兴的最根本保证。

\subsubsection{新时代中国共产党的历史使命}

新时代中国共产党的历史使命,就是统揽伟大斗争、伟大工程、伟大事业、伟大梦想,在全面建成小康社会的基础上全面建成社会主义现代化强国,实现中华民族伟大复兴的中国梦。

伟大梦想是目标,伟大斗争是手段,伟大工程是保障,伟大事业是主题。

其中,起决定性作用的是党的建设伟大工程。

\subsection{坚持党对一切工作的领导}

\subsubsection{党是最高政治领导力量}

\subsubsection{党的领导制度是我国的根本领导制度}

\subsubsection{确保党始终统揽全局协调各方}

\vspace*{30pt}

\begin{center}
  \Large{THE END}
\end{center}

\end{document}