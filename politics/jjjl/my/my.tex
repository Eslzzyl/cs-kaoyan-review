% draft会跳过文档中的所有图片。正式导出时需要删掉draft参数。
\documentclass[12pt, a4paper, oneside]{ctexart}

\usepackage{amsmath}
\usepackage{amssymb}
\usepackage{amsopn}
\usepackage{bm}
\usepackage{graphicx}
\usepackage{mathrsfs}
\usepackage{geometry}
\usepackage{framed}
\usepackage{xcolor}
\usepackage{caption}
\usepackage{listings}
\usepackage{fancyhdr}
\usepackage{booktabs}
\usepackage{makecell}
\usepackage{multirow}
\usepackage{indentfirst}
\usepackage{authblk}
\usepackage{multicol}
% \usepackage{draftwatermark}       % 需要应用水印时取消注释
\usepackage{enumitem}
\usepackage[hidelinks]{hyperref}
\usepackage{tikz}
\usepackage{ulem}
\usetikzlibrary{positioning, shapes.geometric}

% 分栏线宽
\columnseprule=0.4pt

% 定制第二级无序列表的点样式
\setlist[itemize,2]{label=$\diamond$}

% 页边距
\geometry{a4paper, scale=0.8}

\pagestyle{fancy}

% 调整页眉高度,用于去除警告
\setlength{\headheight}{25pt}

\fancyhf{}      % 清空页眉页脚设置
\fancyhead[L] {
    % 工大计算机系logo
    \includegraphics[height=7mm]{../../../share/images/logo1.jpg}
}
\fancyhead[C]{精讲精练-马原部分}
\fancyhead[R]{\leftmark}    % 右侧页眉:当前章标题

% 页脚居中放置页码
\fancyfoot[C]{\thepage}

% 设置章节标题自动编号的格式
\ctexset{
  section/number=\chinese{section},
%   subsection/name={,},
%   subsection/number=\chinese{subsection}
}

% 行距。ctexart默认值为1.3
\linespread{1.2}

\lstset{
  language=c,
  basicstyle=\ttfamily,
  frame=single,
  numbers=left
}

% \SetWatermarkText{Eslzzyl整理}            % 设置水印内容
% \SetWatermarkLightness{0.9}             % 设置水印透明度 0-1
% \SetWatermarkScale{0.8}                   % 设置水印大小 0-1

\renewcommand{\headrulewidth}{1pt}  %页眉线宽,设为0可以去页眉线
\renewcommand{\footrulewidth}{1pt}  %脚注线的宽度

\definecolor{shadecolor}{RGB}{241, 241, 255}

\title{
    \includegraphics[width=0.3\textwidth]{../../../share/images/hfut-badge.pdf}
    
    \vspace{20pt}
    肖秀荣《精讲精练》\\ 之 \\ 马克思主义基本原理
}
\author{Eslzzyl}
\date{\today}

\newcounter{problemname}
\newenvironment{problem}{\begin{shaded}\stepcounter{problemname}\par\noindent\textbf{例题\arabic{problemname}. }}{\end{shaded}\par}
\newenvironment{solution}{\begin{shaded}\par\noindent\textbf{解答:}}{\end{shaded}\par}
% \newenvironment{solution}{\par\noindent\textbf{答案. }}{\par}
% \newenvironment{note}{\par\noindent\textbf{例题\arabic{problemname}的注记. }}{\\\par}
\newenvironment{note}{\par\noindent\textbf{注记. }}{\par}

\begin{document}

\maketitle
\newpage
\tableofcontents
\vspace{20pt}
% 如果在目录处有备注,可以写在这里。

\newpage

\section{导论}

\subsection{马克思主义的创立和发展}

\subsubsection{马克思主义的创立}

\begin{itemize}
  \item 资本主义经济的发展为马克思主义的产生提供了\textbf{经济、社会历史条件}。
  \item \textbf{无产阶级}在反抗资产阶级的斗争中逐步走向自觉,对科学理论的指导提出了强烈的需求,为马克思主义产生提供了\textbf{阶级基础}和\textbf{实践基础}。
  
  19世纪三大工人运动:
  \begin{itemize}
    \item 法国里昂工人起义
    \item 英国宪章运动
    \item 德国西里西亚纺织工人起义
  \end{itemize}

  \item 马、恩的革命实践和对人类文明成果的继承与创新(\textbf{主观条件})
  
  马克思主义的\textbf{直接理论来源}:
  \begin{itemize}
    \item 德国古典哲学
    \item 英国古典政治经济学
    \item 英法空想社会主义
  \end{itemize}

  马克思主义产生的\textbf{自然科学前提}:
  \begin{itemize}
    \item 细胞学说
    \item 能量守恒和转化定律
    \item 生物进化论
  \end{itemize}
\end{itemize}

产生过程:
\begin{itemize}
  \item \textbf{在《德法年鉴》上发表的论文}表明马、恩完成了从唯心主义向唯物主义、从革命民主主义向共产主义的转变,为创立马克思主义奠定了\textbf{思想前提}。
  \item \textbf{《德意志意识形态》}第一次比较系统地阐述了历史唯物主义基本原理。
  \item \textbf{《共产党宣言》}的发表标志着马克思主义的公开问世。
  \item \textbf{《资本论》}(马克思)是“工人阶级的圣经”
  \item \textbf{《反杜林论》}(恩格斯)全面地阐述了马克思主义理论体系。
\end{itemize}

马、恩领导创建的\textbf{世界上第一个无产阶级政党}是\textbf{共产主义者同盟}。

\subsubsection{马克思主义的发展}

\begin{enumerate}
  \item {\bf 列宁的发展}
  
  列宁提出了社会主义革命可能在一国或数国首先发生并取得胜利的论断。

  俄国十月革命的胜利,使科学社会主义\textbf{开始从理论变为现实}(非巴黎公社)。

  \item {\bf 中国化时代化}
  
  毛泽东思想、中国特色社会主义理论体系、习近平新时代中国特色社会主义思想
\end{enumerate}

\subsection{马克思主义的基本特征和当代价值}

\subsubsection{基本特征}

\begin{itemize}
  \item {\bf 科学性}
  
  科学的世界观和方法论基础:辩证唯物主义和历史唯物主义

  \item {\bf 人民性}
  
  人民性是马克思主义的本质属性,是马克思主义最鲜明的品格;人民至上是马克思主义的政治立场。

  归根到底就是一句话:为人类求解放。
  
  \item {\bf 实践性}
  
  这是马克思主义区别于其他理论的\textbf{显著特征}。

  \item {\bf 发展性}
\end{itemize}

马克思主义的基本特征,就是\textbf{科学性}和\textbf{革命性}的统一。人民性、实践性和发展性集中地体现为革命性。

\subsubsection{当代价值}

\begin{itemize}
  \item 观察当今世界变化的认识工具
  \item 指引当代中国发展的行动指南
  \item 引领人类社会进步的科学真理
\end{itemize}

\section{世界的物质性及发展规律}

\subsection{世界的多样性与物质统一性}

\subsubsection{物质及其存在方式}

\begin{enumerate}
  \item {\bf 哲学和它的基本问题}
  
  哲学的基本问题是存在和思维的关系问题。它又包括两个方面:
  \begin{itemize}
    \item 物质和精神何者为第一性、何者为第二性的问题
    \begin{itemize}
      \item 这个不是说哪个重要,这个和价值没有关系
    \end{itemize}
    \item 思维和存在有无同一性的问题(思维能否正确认识存在)
  \end{itemize}

  \begin{figure}[h]
    \centering
    \includegraphics[width=0.9\textwidth]{./images/philosophy.jpg}
    \caption{哲学划分}
  \end{figure}

  \begin{itemize}
    \item 唯物主义总是认同可知论,唯心主义却不总是认同不可知论。
    \item 唯心主义$\neq$形而上学,唯物主义$\neq$辩证法。
    \item 唯心主义又分:
    \begin{itemize}
      \item 主观唯心主义:心、观念、感觉等
      \item 客观唯心主义:理、理念、绝对观念等
    \end{itemize}
  \end{itemize}

  \item {\bf 物质}
  
  \begin{table}[h]
    \centering
    \caption{三种形态的唯物主义对物质的理解}
    \begin{tabular}{|c|c|c|}
      \hline
      形态 & 对物质的理解 & 共同点 \\ \hline
      古代朴素唯物主义 & \makecell[c]{金木水火土等\\ 具体的物质形态} & \multirow{3}{*}{\makecell[c]{从朴素到科学,从片面到比较全面,\\ 但都把世界的本原归结为物质,\\ 主张物质第一性,意识第二性}} \\ \cline{1-2}
      近代形而上学唯物主义 & 原子 & \\ \cline{1-2}
      现代辩证唯物主义 & 客观实在 & \\
      \hline
    \end{tabular}
  \end{table}

  物质的共同特性(唯一特性)是\textbf{客观实在性}。客观实在不一定是看得见摸得着的东西,重点在于它独立于人的意识之外。

  马克思主义物质观的理论意义:
  \begin{itemize}
    \item 略
  \end{itemize}

  \item {\bf 物质的存在方式}
  
  运动是物质的根本属性,物质和运动不可分割。

  运动是绝对的,静止是相对的。

  \item {\bf 物质世界的二重化}
  
  人类改造世界的实践活动使世界发生了二重化,即从自然界中分化出人类社会,从客观世界中分化出主观世界。
\end{enumerate}

\subsubsection{物质和意识的辩证关系}

物质决定意识,意识依赖于物质并反作用于物质。

\begin{enumerate}
  \item {\bf 物质决定意识}
  
  意识的起源:
  \begin{itemize}
    \item 意识是自然界长期发展的产物
    \item 意识又是社会历史发展的产物
    \begin{itemize}
      \item 社会实践特别是劳动在意识的产生和发展中起着决定性的作用。
    \end{itemize}
  \end{itemize}

  意识的本质:意识是特殊的物质——人脑的机能和属性。意识是客观世界的主观映像,是客观内容和主观形式的统一。

  人脑是意识的器官,但不是意识的源泉。意识的源泉是客观世界。

  \item {\bf 意识对物质具有反作用}
  
  表现在:
  \begin{itemize}
    \item 意识活动有目的性和计划性
    \item 意识活动有创造性
    \item 意识有指导实践改造客观世界的作用
    \item 意识具有调控人的行为和生理活动的作用
  \end{itemize}

  \item {\bf 主观能动性和客观规律性的统一}
  
  可出分析题。P17

\end{enumerate}

\subsubsection{世界的物质统一性}

\begin{enumerate}
  \item {\bf 世界的物质统一性原理}
  
  世界是统一的,世界的本原只有一个,就是物质。
  \begin{itemize}
    \item 自然界是物质的。
    \item 人类社会本质上是生产实践基础上形成的物质体系。
    \item 人的意识统一于物质。
  \end{itemize}
  \item {\bf 原理的意义}
  
  \begin{itemize}
    \item {\bf 理论意义}:这原理是辩证唯物主义最基本、最核心的观点,是马克思主义的基石。
    \item {\bf 实践意义}:它是我们从事一切工作的立足点,一切从实际出发是唯物主义一元论的根本要求。
  \end{itemize}

\end{enumerate}

\subsection{事物的普遍联系和变化发展}

\subsubsection{联系和发展的普遍性}

\begin{enumerate}
  \item {\bf 事物的普遍联系}
  
  联系的特点:
  \begin{itemize}
    \item 客观性
    \item 普遍性
    \begin{itemize}
      \item 任何事物内部的不同部分和要素是相互联系的。
      \item 任何事物都不能孤立存在,必然和其他事物相互联系。
      \item 整个世界是相互联系的统一整体。
    \end{itemize}
    \item 多样性
    \begin{itemize}
      \item 直接联系和间接联系
      \item 内部联系和外部联系
      \item 本质联系和非本质联系
      \item 必然联系和偶然联系
    \end{itemize}
    \item 条件性
  \end{itemize}

  \item {\bf 事物的变化发展}
  
  发展的实质是新事物的产生和旧事物的灭亡。

  新生事物是不可战胜的,因为:
  \begin{itemize}
    \item 新事物适应了变化了的环境和条件
    \item 新事物是在旧事物的母体中孕育的
  \end{itemize}

  新旧事物相互区别的根本标志在于是否同历史发展的必然趋势相符合,而不能单凭出现时间的早晚来判断。
\end{enumerate}

\subsubsection{对立统一规律}

\begin{enumerate}
  \item {\bf 唯物辩证法的实质和核心}
  
  对立统一规律,又叫\textbf{矛盾规律},是唯物辩证法的实质和核心,是事物发展的\textbf{根本规律}。

  对立统一规律提供了人们\textbf{认识世界和改造世界的根本方法:矛盾分析方法}。

  是否承认对立统一学说是唯物辩证法和形而上学对立的实质。

  \item {\bf 矛盾的同一性和斗争性及其在事物发展中的作用}
  
  同一性和斗争性是矛盾的两种\textbf{基本属性}。

  什么是同一性:
  \begin{itemize}
    \item 矛盾的对立面相互依存,互为存在。
    \item 矛盾的对立面在一定条件下可以相互转化。
  \end{itemize}

  \begin{itemize}
    \item 矛盾的斗争性是无条件的、绝对的
    \item 矛盾的同一性是有条件的、相对的
  \end{itemize}

  同一性在事物发展中的作用:
  \begin{itemize}
    \item 矛盾双方可以利用对方的发展使自己获得发展。
    \item 矛盾双方可以相互吸取有利于自身的因素,在相互作用中各自得到发展。
    \item 矛盾双方可以向着自己的对立面转化而得到发展,并规定着事物发展的方向。
  \end{itemize}

  斗争性在事物发展中的作用:
  \begin{itemize}
    \item 促进矛盾双方力量的变化,造成事物的量变,为对立面的转化、事物的质变创造条件。
    \item 斗争促使矛盾双方地位或性质转化,实现事物的质变。
  \end{itemize}

  矛盾的斗争性和同一性共同作用,推动了事物的发展。在不同条件下,二者所处的地位可能不同。

  方法论意义:
  \begin{itemize}
    \item 在分析和解决矛盾时,必须从对立中把握同一,从同一中把握对立。
    \item 正确把握\textbf{和谐}对事物发展的作用。
    \begin{itemize}
      \item \textbf{和谐}是矛盾的一种特殊表现形式,体现着矛盾双方的相互依存、相互促进、共同发展。和谐是相对的、有条件的。
    \end{itemize}
  \end{itemize}

  事物发展的根本原因不在事物外部,而在事物内部的矛盾性。

  \item {\bf 矛盾的普遍性和特殊性及其相互关系}
  
  \begin{itemize}
    \item {\bf 普遍性}:是矛盾的共性。即:一切事物都是对立统一的,矛盾是事物的普遍本质。
    \item {\bf 特殊性}:是矛盾的个性。
    \begin{itemize}
      \item 不同事物的矛盾各有其特点
      \item 同一事物的矛盾在不同发展过程和不同发展阶段各有其特点
      \item 构成事物的\textbf{诸多矛盾}以及每一矛盾的\textbf{不同方面}各有不同的性质、地位和作用。
      \begin{itemize}
        \item 事物由多种矛盾构成,可分为主要矛盾和次要矛盾。
        \item 每种矛盾又有主要方面和次要方面。
        \item 事物的性质由主要矛盾的主要方面规定。
        \item 在实际工作中,要坚持“两点论”和“重点论”的统一。
      \end{itemize}
    \end{itemize}
  \end{itemize}

  可以说共性寓于个性之中,但不能说个性寓于共性之中,因为个性比共性更丰富。

  矛盾的普遍性和特殊性的辩证关系原理是马克思主义的普遍整理同各国的具体实际相结合的哲学基础。
\end{enumerate}

\subsubsection{量变质变规律和否定之否定规律}

\begin{enumerate}
  \item {\bf 量变质变规律}
  
  事物的质、量、度:
  \begin{itemize}
    \item {\bf 质}:是事物成为自身并区别于其他事物的内在规定性。
    \item {\bf 量}:是事物的规模、程度、速度等可以用数量关系表示的规定性。
    \item {\bf 度}:是保持事物的稳定性的数量界限。超出度的范围,此物就转化为他物。
  \end{itemize}

  区分量变和质变的\textbf{根本标志}是事物的变化是否超出度。

  量变和质变的辩证关系:
  \begin{itemize}
    \item 量变是质变的重要准备
    \item 质变是量变的必然结果
    \item 量变和质变相互渗透
  \end{itemize}

  量变质变规律体现了事物发展的渐进性和飞跃性的统一。

  \item {\bf 否定之否定规律}
  
  任何事物内部都有肯定因素和否定因素。
  \begin{itemize}
    \item 肯定因素是维持现存事物存在的因素
    \item 否定因素是促使现存事物灭亡的因素
  \end{itemize}

  在新事物取代旧事物的过程中,辩证的否定是决定性的环节。辩证否定是事物的\textbf{自我否定},不是相互否定。

  否定之否定规律揭示了事物发展的方向和道路——前进性和曲折性的统一。
\end{enumerate}

\subsubsection{联系和发展的基本环节}

\begin{enumerate}
  \item {\bf 内容与形式}(构成要素与表现方式)
  
  内容决定形式,形式反作用于内容。

  形式对内容的反作用表明,形式具有\textbf{相对独立性}。

  在内容与形式的矛盾运动中,内容较为易变,形式则较为稳定。

  \item {\bf 本质与现象}(内在联系与外在表现)
  
  关系:
  \begin{itemize}
    \item 一方面,本质与现象是相互区别的;
    \begin{itemize}
      \item 现象是个别的、具体的,本质是一般的、普遍的。
      \item 现象是多变易逝的,本质是相对稳定的。
      \item 现象可以被人的感官直接感知,本质则靠人的理性思维才能把握。
    \end{itemize}
    \item 另一方面,本质与现象又是相互依存的。
    \begin{itemize}
      \item 本质决定现象
      \item 现象表现本质
      \item 不表现为现象的本质和不表现本质的现象都是不存在的。
    \end{itemize}
  \end{itemize}

  现象有真假之分,真象和假象都是客观的,而错觉是主观的。真象$\neq$真相。

  \item {\bf 原因与结果}
  
  因果联系的特点:有时间顺序。总是原因在前结果在后。

  原因与结果的关系是复杂多样的,有一因多果、同因异果、一果多因、异因同果、多因多果、复合因果等。

  \item {\bf 必然与偶然}
  
  关系:
  \begin{itemize}
    \item 一方面,二者是有区别的;
    \begin{itemize}
      \item 必然产生于事物内部的根本矛盾,偶然产生于非根本矛盾和外部条件。
      \item 表现形式不同
      \item 作用不同。必然居于支配地位,决定事物发展的方向;偶然居于从属地位,对事物的发展起到促进或延缓作用。
    \end{itemize}
    \item 另一方面,二者又是统一的。
    \begin{itemize}
      \item 二者相互依存:没有脱离偶然的必然,也没有脱离必然的偶然。
      \item 一定条件下二者可以相互转化。
    \end{itemize}
  \end{itemize}
  \item {\bf 现实与可能}
  
  关系:
  \begin{itemize}
    \item 相互区别(对立)
    \item 相互转化(统一)
  \end{itemize}
\end{enumerate}

\subsection{唯物辩证法}

唯物辩证法是认识世界和改造世界的根本方法

\subsubsection{唯物辩证法的本质特征和认识功能}

\begin{enumerate}
  \item {\bf 唯物辩证法本质是批判的革命的}
  
  它认为,一切事物都在发生、发展和灭亡的过程中。没有最终的东西、绝对的东西、神圣的东西。

  \item {\bf 客观辩证法和主观辩证法的统一}
  
  \begin{itemize}
    \item {\bf 客观辩证法}:客观事物或客观存在的辩证法
    \item {\bf 主观辩证法}:人类认识和思维活动的辩证法。也叫概念辩证法。
  \end{itemize}、
  
  矛盾分析方法是对立统一规律在方法论上的体现,在唯物辩证法的方法论体系中居于核心地位,是我们认识事物的根本方法。

  马克思主义最本质的东西,马克思主义活的灵魂,就在于具体地分析具体的情况。

\end{enumerate}

\subsubsection{学习唯物辩证法}

在实际工作中不断增强:
\begin{itemize}
  \item 辩证思维能力
  \begin{itemize}
    \item {\bf 归纳与演绎}
    \begin{itemize}
      \item 归纳:从个别到一般
      \item 演绎:从一般到个别
    \end{itemize}
    \item {\bf 分析与综合}
    \item 二者是两种相反的思维方法,分析是综合的基础,综合是分析的完成。
    \item {\bf 抽象与具体}
    \item {\bf 逻辑与历史相统一}
  \end{itemize}
  \item 历史思维能力
  \item 系统思维能力
  \item 战略思维能力
  \item 底线思维能力
  \item 创新思维能力
\end{itemize}

\section{实践与认识及其发展规律}

认识论:
\begin{itemize}
  \item 两对关系
  \begin{itemize}
    \item 实践与认识的关系
    \item 真理与价值的关系
  \end{itemize}
  \item 两个规律
  \begin{itemize}
    \item 认识发展的规律
    \item 真理发展的规律
  \end{itemize}
  \item 一个统一
  \begin{itemize}
    \item 认识与实践的统一
  \end{itemize}
\end{itemize}

\subsection{实践与认识}

实践性是马克思主义区别于其他理论的\textbf{根本特征},实践的观点是马克思主义的\textbf{基本观点}。

\subsubsection{科学的实践观及其意义}

略

\subsubsection{实践的本质与基本结构}

\begin{enumerate}
  \item {\bf 实践的本质}
  
  实践是人类能动地改造世界的社会性的物质活动。

  三个基本特征:
  \begin{itemize}
    \item {\bf 客观实在性}:实践高于认识
    \item {\bf 自觉能动性}
    \item {\bf 社会历史性}
  \end{itemize}

  \item {\bf 实践的基本结构}
  
  实践活动的三要素:
  \begin{itemize}
    \item 主体
    
    实践主体的能力:
    \begin{itemize}
      \item 自然能力
      \item 精神能力
      \begin{itemize}
        \item 知识性因素(首要能力)
        \item 非知识性因素
      \end{itemize}
    \end{itemize}
    \item 客体
    \item 实践中介
    \begin{itemize}
      \item 作为人的肢体延长、感官延伸、体能放大的物质性工具系统
      \item 语言符号系统
    \end{itemize}
  \end{itemize}

  实践的主体和客体相互作用的关系:
  \begin{itemize}
    \item 实践关系(最根本)
    \item 认识关系
    \item 价值关系
  \end{itemize}

  \begin{itemize}
    \item {\bf 主体客体化}:人通过实践使自己的本质力量作用于客体,形成了世界上本来不存在的对象物。
    \item {\bf 客体主体化}:如把物质工具作为自己身体器官的延长。
  \end{itemize}

  \item {\bf 实践形式的多样性}
  
  实践的三种基本类型:
  \begin{itemize}
    \item 物质生产实践(最基本的实践活动)
    \item 社会政治实践
    \item 科学文化实践
  \end{itemize}

  现代新的实践形式:虚拟实践

  \item {\bf 实践对认识的决定作用}
  
  实践对认识的决定作用体现在:
  \begin{itemize}
    \item 实践是认识的来源(并不否认学习间接经验的重要性)
    \item 实践是认识发展的动力
    \item 实践是认识的目的
    \item 实践是检验认识真理性的唯一标准
  \end{itemize}

  实践也要受认识的指导。

  实践的观点是辩证唯物主义认识论的第一和基本的观点。
\end{enumerate}

\subsubsection{认识的本质和过程}

\begin{enumerate}
  \item {\bf 认识的本质}
  
  认识是主体在实践基础上对客体的能动反映。

  两条根本对立的认识路线:
  \begin{itemize}
    \item {\bf 唯物主义反映论}:认识是主体对客体的反映
    \begin{itemize}
      \item 旧唯物主义:认为认识是消极被动地反映外界对象,类似于照镜子。离开实践考察认识问题,认为认识是一次性完成的。
      \item 辩证唯物主义:建立在实践基础上的能动的反映论
    \end{itemize}
    \item {\bf 唯心主义先验论}
    \begin{itemize}
      \item 主观唯心主义:认识是主观自生的
      \item 客观唯心主义:认识是上帝的启示或绝对精神的产物
    \end{itemize}
  \end{itemize}
  \item {\bf 从实践到认识}
  
  从实践到认识,是认识过程中的\textbf{第一次飞跃}。

  感性认识和理性认识:
  \begin{itemize}
    \item {\bf 感性认识}
    \begin{itemize}
      \item 通过感官直接感受到
      \item 包括感觉、知觉和表象三种形式
      \item 是认识的低级阶段
      \item 特点:直接性、具体性
    \end{itemize}
    \item {\bf 理性认识}
    \begin{itemize}
      \item 运用抽象思维
      \item 包括概念、判断、推理三种形式
      \item 是认识的高级阶段
      \item 特点:抽象性、间接性
    \end{itemize}
  \end{itemize}

  二者的联系:
  \begin{itemize}
    \item 理性认识依赖于感性认识
    \item 感性认识有待于发展和深化为理性认识
    \item 感性认识和理性认识相互渗透、相互包含
  \end{itemize}

  \begin{itemize}
    \item 片面强调感性认识,会走向\textbf{经验论},在实际工作中表现为\textbf{经验主义}。
    \item 片面强调理性认识,会走向\textbf{唯理论},在实际工作中表现为\textbf{教条主义}。
  \end{itemize}

  \item {\bf 从认识到实践}
  
  从认识到实践,是认识过程中的\textbf{第二次飞跃},比从实践到认识的过程更重要。

\end{enumerate}

\subsubsection{实践与认识的辩证运动及其规律}

从实践到认识、从认识到实践,如此反复,是人类认识运动的辩证发展过程,也是人类认识运动的基本规律。

\subsection{真理与价值}

\subsubsection{真理的客观性、绝对性和相对性}

\begin{enumerate}
  \item {\bf 真理的客观性}
  
  真理是客观的,凡真理都是客观真理。客观性是真理的\textbf{本质属性}。

  真理的内容是客观的,形式是主观的。真理是一种认识。

  真理的客观性决定了真理的\textbf{一元性},即:同一条件下对于特定的认识客体的真理性认识只有一个,不因主体认识的差别和变化而改变。(不能说“对...来说是真理”)

  \item {\bf 真理的绝对性和相对性}
  
  \begin{itemize}
    \item {\bf 绝对性}
    \begin{itemize}
      \item 任何真理都包含不依赖人和人的意识的客观内容,都同谬误有原则的界限。
      \item 人类认识按其本质来说,能够正确认识物质世界,即世界是可知的。
    \end{itemize}
    \item {\bf 相对性}:人在一定条件下对真理的认识总是有限度的、不完整的。
    \begin{itemize}
      \item 任何真理都只是对客观世界的某一阶段、某一部分的正确认识,\textbf{有待扩展}。
      \item 任何真理都只是对客观对象的一定方面、一定层次和一定程度的正确认识,\textbf{有待深化}。
    \end{itemize}
  \end{itemize}

  二者的辩证关系:
  \begin{itemize}
    \item 二者相互依存
    \item 二者相互包含
    \begin{itemize}
      \item 毛泽东:“无数相对的真理之总和,就是绝对的真理”。
    \end{itemize}
  \end{itemize}

  二者根源于人类认识世界能力的\textbf{无限性}与\textbf{有限性}、\textbf{绝对性}与\textbf{相对性}的矛盾。将人类视为一个整体来看,是能够认识无限的物质世界的,但具体到每个人,总是要受到各种限制。

  \begin{itemize}
    \item 片面夸大绝对性,即绝对主义,将导致教条主义和思想僵化。
    \item 片面夸大相对性,即相对主义,将导致不可知论和诡辩论。
  \end{itemize}

  已经确定的真理是不能被推翻的,只能进行完善。

  \item {\bf 真理与谬误}
  
  对立统一关系:
  \begin{itemize}
    \item 真理和谬误的对立是绝对的,二者存在着原则界限。
    \item 绝对的对立是在一定范围内成立的,超出这个范围,它们就会相互转化。这是真理和谬误的统一性。
  \end{itemize}
\end{enumerate}

\subsubsection{真理的检验标准}

实践的直接现实性,是它能够成为检验真理的唯一标准的主要根据。

如果把主观的东西当作检验真理的标准,就是用认识检验认识,就属于主观真理标准论。

逻辑证明可以在检验真理的过程中起到补充作用。

\subsubsection{真理与价值的辩证统一}

\begin{enumerate}
  \item {\bf 价值}
  
  价值本质是主体和客体之间的一种特定的关系。
  
  价值的基本特性:
  \begin{itemize}
    \item {\bf 主体性}($\neq$主观性)
    \item {\bf 客观性}
    \item {\bf 多维性}
    \item {\bf 社会历史性}
  \end{itemize}

  \item {\bf 价值评价及其特点}
  
  基本特点:
  \begin{itemize}
    \item 评价以主客体之间的价值关系为认识对象
    \item 评价结果与评价主体直接相关,受主体意志的影响
    \item 评价结果的正确与否依赖于对客体状况和主体需要的认识
    \item 价值评价有科学和非科学之别
  \end{itemize}

  \item {\bf 真理与价值在实践中的辩证统一}
  
  实践的
  \begin{itemize}
    \item 真理尺度:人们必须遵循正确的规律进行实践
    \item 价值尺度:人们都是按照自己的尺度和需要进行实践的
  \end{itemize}

  任何成功的实践都是真理尺度和价值尺度的统一。

\end{enumerate}

\subsection{认识世界和改造世界}

\subsubsection{认识世界的根本目的在于改造世界}

认识世界和改造世界是人类创造历史的两种\textbf{基本活动}。

认识世界和改造世界是相互依赖、相互制约的辩证统一关系。二者统一的基础是\textbf{实践}。

改造世界又可分为
\begin{itemize}
  \item 改造客观世界
  \begin{itemize}
    \item 改造自然界
    \item 改造人类社会
  \end{itemize}
  \item 改造主观世界
  \begin{itemize}
    \item 提高人的认识能力
    \item 丰富人的情感世界
    \item 提升人的意识品质,等等
    \item 核心是改造世界观。
  \end{itemize}
\end{itemize}

\textbf{主观和客观的矛盾}是人类认识和实践过程中的\textbf{基本矛盾},也是人类认识世界和改造世界的\textbf{根本动力}。

自由和必然:
\begin{itemize}
  \item {\bf 自由}是人在活动中通过认识和利用必然所表现出来的一种\textbf{自觉自主的状态}。
  \item {\bf 必然}性就是规律性,是指不依赖人的意志而存在的固有的客观规律。
\end{itemize}

\textbf{认识必然、争取自由},是人类认识世界和改造世界的\textbf{根本目标}。

\subsubsection{一切从实际出发,实事求是}

一切从实际出发是马克思主义认识论的根本要求。

实事求是中国共产党思想路线的核心。

\section{人类社会及其发展规律}

\subsection{人类社会的存在与发展}

\subsubsection{社会存在与社会意识}

哲学问题在历史观中的贯彻和表现就是历史观的基本问题,即社会存在和社会意识的关系问题。

\begin{enumerate}
  \item {\bf 两种根本对立的历史观}
  
  即唯物史观和唯心史观。

  社会存在和社会意识,主张谁决定谁,是两种史观的根本分野。

  \item {\bf 社会存在}
  
  社会存在是社会生活的物质方面,主要包括:
  \begin{itemize}
    \item {\bf 自然地理环境}
    \item {\bf 人口因素}
    \item {\bf 物质生产方式}
    
    前两者都只是影响因素,对社会发展起着延缓或加速的作用,但不能决定社会的性质和社会形态的更替。

    物质生产方式是生产力和生产关系的统一体,是\textbf{社会历史发展的决定性力量}。
  \end{itemize}

  \item {\bf 社会意识}
  
  可分为意识形态和非意识形态。

  意识形态有鲜明的阶级性。

  \begin{figure}[h]
    \centering
    \includegraphics[width=0.9\textwidth]{./images/social-consciousness.jpg}
  \end{figure}

  \item {\bf 社会存在和社会意识的辩证关系}

  社会存在决定社会意识,社会意识反映社会存在,并反作用于社会存在。

  社会意识有相对独立性,例如经济水平发展高的国家,其社会意识的发展水平未必都是最高的。

  社会意识的能动作用是通过指导人们的实践活动实现的。

\end{enumerate}

\subsubsection{社会基本矛盾及其运动规律}

\begin{enumerate}
  \item {\bf 生产力和生产关系}
  
  \begin{itemize}
    \item {\bf 生产力}
    
    有复杂的系统结构,包括:
    \begin{itemize}
      \item 劳动资料(劳动手段),其中最重要的是生产工具。生产工具是生产力发展的客观尺度,是区分社会经济时代的物质标志。
      \item 劳动对象,可以是一切自然物质。从侧面反映了生产力的发展水平。
      
      劳动资料和劳动对象合称生产资料。

      \item 劳动者,是生产力中最活跃的因素。
    \end{itemize}
    \item {\bf 生产关系}
    
    生产关系是社会关系中最基本的关系。包括:
    \begin{itemize}
      \item 生产资料所有制关系(最基本,决定性)
      \item 生产中人与人的关系
      \item 产品分配关系
    \end{itemize}
  \end{itemize}

  生产关系一定要适合生产力状况的规律,这是\textbf{人类社会发展的一个基本规律}。

  生产力决定生产关系,生产关系有反作用于生产力。

  \item {\bf 经济基础与上层建筑}

  \begin{itemize}
    \item {\bf 经济基础}
    
    是指由社会一定发展阶段的生产力所决定的生产关系的总和。

    一个社会阶段可能有多种生产关系,但决定社会性质的是其中占支配地位的生产关系。

    \item {\bf 上层建筑}
    
    包括:
    \begin{itemize}
      \item 意识形态(观念上层建筑),包括政治法律思想、道德、艺术、宗教、哲学等
      \item 政治法律制度及设施和政治组织(政治上层建筑),包括国家政治制度、立法司法制度和行政制度,以及国家的各种机构。
    \end{itemize}

    其中,政治上层建筑居主导地位,国家政权是核心。

    \begin{figure}[h]
      \centering
      \includegraphics[width=0.9\textwidth]{./images/superstructure.jpg}
    \end{figure}
  \end{itemize}

  上层建筑一定要适应经济基础状况,是\textbf{人类社会发展的又一个基本规律}。
\end{enumerate}

\subsubsection{人类普遍交往与世界历史的形成发展}

\begin{enumerate}
  \item {\bf 交往}
  
  可以把交往划分为物质交往和精神交往。

  作用:促进生产力的发展;促进社会关系的进步;促进文化的发展与传播;促进人的全面发展。

  \item {\bf 世界历史的形成与发现}
  
  世界历史是指各民族、国家通过普遍交往、打破孤立隔绝的状态,进入相互依存、相互联系的世界整体化的历史。

  \begin{itemize}
    \item 生产方式的发展变革是世界历史形成和发展的基础。
    \item 普遍交往是世界历史的基本特征。
  \end{itemize}
\end{enumerate}

\subsubsection{社会进步与社会形态更替}

\begin{enumerate}
  \item {\bf 社会进步与人的发展}
  
  人的发展,最根本的是人的自由程度的提高。人的发展程度构成了社会进步的重要标志。

  \item {\bf 社会形态的内涵}
  
  社会形态就是社会制度。

  \item {\bf 社会形态更替的前进性与曲折性}
  
  社会主义是人类历史迄今最进步的社会形态。
\end{enumerate}

\subsubsection{文明及其多样性}

文明是标志社会进步程度的范畴,反映了人类社会实践活动的积极成果。

\subsection{社会历史发展的动力}

\subsubsection{社会基本矛盾}

\begin{itemize}
  \item 生产力和生产关系的矛盾(最基本)
  \item 经济基础与上层建筑的矛盾
\end{itemize}

其中,经济基础是生产关系的总和,因此基本矛盾实际上包括三个方面:生产力、生产关系、上层建筑。

社会\textbf{基本}矛盾贯穿社会发展的始终,而社会\textbf{主要}矛盾可以发生阶段性变化。

社会基本矛盾是历史发展的\textbf{根本动力}。

生产方式是生产力和生产关系的统一,是历史发展的\textbf{决定力量}。其中生产力是\textbf{最终决定力量}。

一些“动力”:电子书P58

\subsubsection{阶级斗争和社会革命}

\begin{enumerate}
  \item {\bf 阶级和阶级斗争}
  
  阶级:与特定的生产关系相联系的、在经济上处于不同地位的社会集团或人群共同体。

  生产资料占有关系的不同,是划分阶级的基础。

  \item {\bf 阶级社会发展的直接动力}
  
  阶级斗争是社会基本矛盾在阶级社会种的直接表现,是阶级社会发展的\textbf{直接动力}。

  \item {\bf 社会革命}
  
  社会革命的根本问题是国家政权问题。

  社会革命根源于社会基本矛盾的尖锐化。

  马克思主义不否认\textbf{改良}作为革命的一种补充手段所起的作用,但反对改良主义。

  \item {\bf 改革}
  
  改革是\textbf{同一种社会形态}发展过程中的量变和部分质变,是推动社会发展的又一重要动力。

  改革是社会主义的自我完善和自我发展。
\end{enumerate}

\subsubsection{科学技术}

科学技术是社会发展的重要动力。

\subsubsection{文化}

文化是推动社会发展的重要力量。

\subsection{人民群众在历史发展中的作用}

\subsubsection{人民群众是历史的创造者}

\begin{enumerate}
  \item {\bf 群众史观和英雄史观}
  
  唯物史观和唯心史观,在历史的创造者问题上表现为群众史观和英雄史观的对立。

  \item {\bf 唯物史观考察历史创造者问题的方法论原则}
  
  人的本质是社会属性,而不是自然属性。

  人的本质是变化发展的,而不是永恒不变的。

  人与历史的关系:
  \begin{itemize}
    \item 人类与历史
    \item 群体与历史
    \item 个体与历史
  \end{itemize}

  社会历史的创造者在社会\textbf{内部}。

  \item {\bf 人民群众在创造历史过程中的决定性作用}
  
  人民群众中最稳定的主体部分始终是从事物质资料生产的劳动群众。

  人民群众是社会历史实践的主体,是历史的创造者。

  \item {\bf 无产阶级政党的群众路线}
  
  \textbf{群众路线}:一切为了群众,一切依靠群众,从群众中来,到群众中去。
\end{enumerate}

\subsubsection{个人在社会历史中的作用}

\begin{figure}[h]
  \centering
  \includegraphics[width=0.9\textwidth]{./images/person-in-social-history.jpg}
\end{figure}

\subsubsection{群众、阶级、政党、领袖的关系}

群众是划分为阶级的,阶级通常是由政党领导的,政党是领袖主持的。

\section{资本主义的本质和规律}

\subsection{商品经济和价值规律}

\subsubsection{自然经济}

自然经济特点是自给自足,以使用价值为生产目的。

自然经济是原始社会、奴隶社会和封建社会的基本经济形式。

\subsubsection{商品经济产生的历史条件}

商品经济以交换为生产目的。交换体现了商品经济的本质。

历史条件:
\begin{itemize}
  \item 社会分工的存在
  \item 生产资料和劳动产品属于不同的所有者
\end{itemize}

\subsubsection{商品的二因素}

即使用价值和价值。商品是使用价值和价值的矛盾统一体。

\begin{enumerate}
  \item {\bf 使用价值}
  
  反映的是人与自然之间的物质关系,是商品的自然属性,是一切劳动产品所共有的属性。

  使用价值构成社会财富的物质内容。

  \item {\bf 价值}
  
  是商品的社会属性。

  任何有用物品都有使用价值,但只有商品才有价值。

  价值在本质上体现了生产者之间的一定的社会关系。

  \item {\bf 价值和使用价值的对立统一关系}
  
  \begin{itemize}
    \item {\bf 对立性}:对于交换双方来讲,价值和使用价值是相互排斥的,二者不可兼得。
    \item {\bf 统一性}:作为商品,必须同时具有价值和使用价值两个因素。价值寓于使用价值之中。
  \end{itemize}
\end{enumerate}

\subsubsection{生产商品的劳动的二重性}

\begin{enumerate}
  \item {\bf 具体劳动和抽象劳动}
  
  具体劳动形成商品的使用价值,抽象劳动形成商品的价值。

  抽象劳动是价值的唯一源泉;具体劳动是使用价值(也即社会财富)的源泉,但不是唯一源泉。

  \item {\bf 具体劳动和抽象劳动的对立统一关系}
  
  \begin{itemize}
    \item 具体劳动反映的是人与自然的关系,它是劳动的自然属性
    \item 抽象劳动反映的是商品生产者的社会关系,它是劳动的社会属性。
  \end{itemize}
\end{enumerate}

\subsubsection{商品价值量的决定}

\begin{enumerate}
  \item {\bf 社会必要劳动时间}
  
  决定商品价值量的,是社会必要劳动时间。

  社会必要劳动时间是指在现有的社会正常的生产条件下,在社会平均的劳动熟练程度和劳动强度下制造某种使用价值所需要的劳动时间。

  \item {\bf 商品的价值量和劳动生产率的关系}
  
  劳动生产率指的是劳动者生产使用价值的速率。

  商品的价值量和社会必要劳动时间成正比,和劳动生产率成反比。

  一些影响劳动生产率的因素:
  \begin{itemize}
    \item 劳动者的平均熟练程度
    \item 科学技术的发展水平及其在生产中的应用程度
    \item 生产过程的社会结合
    \item 生产资料的规模
    \item 自然条件,等等
  \end{itemize}

  \begin{table}[h]
    \centering
    \caption{商品的价值量与劳动生产率的关系}
    \begin{tabular}{|c|c|c|}
      \hline
      & 劳动生产率提高 & 劳动生产率降低 \\ \hline
      单位时间内生产的价值量 & 不变 & 不变 \\ \hline
      \makecell[c]{单位时间内生产的使用价值量\\ (即商品数量)} & 增加 & 减少 \\ \hline
      单个商品中包含的价值量 & 减少 & 增加\\ \hline
      单个商品所耗费的劳动时间 & 减少 & 增加 \\
      \hline
    \end{tabular}
  \end{table}

  无论生产力怎样变化,同一劳动在同样的时间内提供的价值量总是相同的。

  形成商品价值量的劳动,以简单劳动为尺度。复杂劳动等于多倍的简单劳动。
\end{enumerate}

\subsubsection{货币}

货币是在长期交换过程中形成的固定充当一般等价物的商品,体现商品生产者之间的社会经济关系。

\begin{table}[h]
  \centering
  \caption{货币的五种基本职能}
  \begin{tabular}{|c|c|}
    \hline
    价值尺度 & 为商品定价,只需要观念的货币 \\ \hline
    流通手段 & 一手交钱,一手交货 \\ \hline
    贮藏手段 & 退出流通领域,必须是足值的金属货币 \\ \hline
    支付手段 & \textbf{延期}支付形式 \\ \hline
    世界货币 & 前面的各种职能在世界范围内的应用 \\
    \hline
  \end{tabular}
\end{table}

前两个是最基本职能。

货币的出现,有利于解决商品交换的困难,促进了商品经济的发展。但是,货币的出现并没有也不可能解决商品经济的基本矛盾,即私人劳动和社会劳动的矛盾,反而使矛盾扩大和加深了。

\subsubsection{价值规律及其作用}

价值规律:商品的价值量由生产商品的社会必要劳动时间决定,商品交换以价值量为基础,按照等价交换的原则进行。

价值规律表现为:商品的价格围绕商品的价值自发波动。

价值规律的作用和引起的消极后果:
\begin{itemize}
  \item 自发地调节生产资料和劳动力在社会各生产部门之间的分配比例——导致社会资源浪费
  \item 自发地刺激社会生产力的发展——阻碍技术进步
  \item 自发地调节社会收入的分配——导致收入两极分化
\end{itemize}

\subsubsection{商品经济的基本矛盾}

商品经济以私有制为基础。在商品经济中,生产者的劳动有两重性:
\begin{itemize}
  \item 具有社会性质的社会劳动,由社会分工决定
  \item 具有私人性质的私人劳动,由生产资料私有制决定
\end{itemize}



\end{document}