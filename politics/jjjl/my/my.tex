% draft会跳过文档中的所有图片。正式导出时需要删掉draft参数。
\documentclass[12pt, a4paper, oneside]{ctexart}

\usepackage{amsmath}
\usepackage{amssymb}
\usepackage{amsopn}
\usepackage{bm}
\usepackage{graphicx}
\usepackage{mathrsfs}
\usepackage{geometry}
\usepackage{framed}
\usepackage{xcolor}
\usepackage{caption}
\usepackage{listings}
\usepackage{fancyhdr}
\usepackage{booktabs}
\usepackage{makecell}
\usepackage{multirow}
\usepackage{indentfirst}
\usepackage{authblk}
\usepackage{multicol}
% \usepackage{draftwatermark}       % 需要应用水印时取消注释
\usepackage{enumitem}
\usepackage[hidelinks]{hyperref}
\usepackage{tikz}
\usepackage{ulem}
\usetikzlibrary{positioning, shapes.geometric}

% 分栏线宽
\columnseprule=0.4pt

% 定制第二级无序列表的点样式
\setlist[itemize,2]{label=$\diamond$}

% 页边距
\geometry{a4paper, scale=0.8}

\pagestyle{fancy}

% 调整页眉高度,用于去除警告
\setlength{\headheight}{25pt}

\fancyhf{}      % 清空页眉页脚设置
\fancyhead[L] {
    % 工大计算机系logo
    \includegraphics[height=7mm]{../../../share/images/logo1.jpg}
}
\fancyhead[C]{精讲精练-马原部分}
\fancyhead[R]{\leftmark}    % 右侧页眉:当前章标题

% 页脚居中放置页码
\fancyfoot[C]{\thepage}

% 设置章节标题自动编号的格式
\ctexset{
  section/number=\chinese{section},
%   subsection/name={,},
%   subsection/number=\chinese{subsection}
}

% 行距。ctexart默认值为1.3
\linespread{1.2}

\lstset{
  language=c,
  basicstyle=\ttfamily,
  frame=single,
  numbers=left
}

% \SetWatermarkText{Eslzzyl整理}            % 设置水印内容
% \SetWatermarkLightness{0.9}             % 设置水印透明度 0-1
% \SetWatermarkScale{0.8}                   % 设置水印大小 0-1

\renewcommand{\headrulewidth}{1pt}  %页眉线宽,设为0可以去页眉线
\renewcommand{\footrulewidth}{1pt}  %脚注线的宽度

\definecolor{shadecolor}{RGB}{241, 241, 255}

\title{
    \includegraphics[width=0.3\textwidth]{../../../share/images/hfut-badge.pdf}
    
    \vspace{20pt}
    肖秀荣《精讲精练》\\ 之 \\ 马克思主义基本原理
}
\author{Eslzzyl}
\date{\today}

\newcounter{problemname}
\newenvironment{problem}{\begin{shaded}\stepcounter{problemname}\par\noindent\textbf{例题\arabic{problemname}. }}{\end{shaded}\par}
\newenvironment{solution}{\begin{shaded}\par\noindent\textbf{解答:}}{\end{shaded}\par}
% \newenvironment{solution}{\par\noindent\textbf{答案. }}{\par}
% \newenvironment{note}{\par\noindent\textbf{例题\arabic{problemname}的注记. }}{\\\par}
\newenvironment{note}{\par\noindent\textbf{注记. }}{\par}

\begin{document}

\maketitle
\newpage
\tableofcontents
\vspace{20pt}
% 如果在目录处有备注,可以写在这里。

\newpage

\section{导论}

\subsection{马克思主义的创立和发展}

\subsubsection{马克思主义的创立}

\begin{itemize}
  \item 资本主义经济的发展为马克思主义的产生提供了\textbf{经济、社会历史条件}。
  \item \textbf{无产阶级}在反抗资产阶级的斗争中逐步走向自觉,对科学理论的指导提出了强烈的需求,为马克思主义产生提供了\textbf{阶级基础}和\textbf{实践基础}。
  
  19世纪三大工人运动:
  \begin{itemize}
    \item 法国里昂工人起义
    \item 英国宪章运动
    \item 德国西里西亚纺织工人起义
  \end{itemize}

  \item 马、恩的革命实践和对人类文明成果的继承与创新(\textbf{主观条件})
  
  马克思主义的\textbf{直接理论来源}:
  \begin{itemize}
    \item 德国古典哲学
    \item 英国古典政治经济学
    \item 英法空想社会主义
  \end{itemize}

  马克思主义产生的\textbf{自然科学前提}:
  \begin{itemize}
    \item 细胞学说
    \item 能量守恒和转化定律
    \item 生物进化论
  \end{itemize}
\end{itemize}

产生过程:
\begin{itemize}
  \item \textbf{在《德法年鉴》上发表的论文}表明马、恩完成了从唯心主义向唯物主义、从革命民主主义向共产主义的转变,为创立马克思主义奠定了\textbf{思想前提}。
  \item \textbf{《德意志意识形态》}第一次比较系统地阐述了历史唯物主义基本原理。
  \item \textbf{《共产党宣言》}的发表标志着马克思主义的公开问世。
  \item \textbf{《资本论》}(马克思)是“工人阶级的圣经”
  \item \textbf{《反杜林论》}(恩格斯)全面地阐述了马克思主义理论体系。
\end{itemize}

马、恩领导创建的\textbf{世界上第一个无产阶级政党}是\textbf{共产主义者同盟}。

\subsubsection{马克思主义的发展}

\begin{enumerate}
  \item {\bf 列宁的发展}
  
  列宁提出了社会主义革命可能在一国或数国首先发生并取得胜利的论断。

  俄国十月革命的胜利,使科学社会主义\textbf{开始从理论变为现实}(非巴黎公社)。

  \item {\bf 中国化时代化}
  
  毛泽东思想、中国特色社会主义理论体系、习近平新时代中国特色社会主义思想
\end{enumerate}

\subsection{马克思主义的基本特征和当代价值}

\subsubsection{基本特征}

\begin{itemize}
  \item {\bf 科学性}
  
  科学的世界观和方法论基础:辩证唯物主义和历史唯物主义

  \item {\bf 人民性}
  
  人民性是马克思主义的本质属性,是马克思主义最鲜明的品格;人民至上是马克思主义的政治立场。

  归根到底就是一句话:为人类求解放。
  
  \item {\bf 实践性}
  
  这是马克思主义区别于其他理论的\textbf{显著特征}。

  \item {\bf 发展性}
\end{itemize}

马克思主义的基本特征,就是\textbf{科学性}和\textbf{革命性}的统一。人民性、实践性和发展性集中地体现为革命性。

\subsubsection{当代价值}

\begin{itemize}
  \item 观察当今世界变化的认识工具
  \item 指引当代中国发展的行动指南
  \item 引领人类社会进步的科学真理
\end{itemize}

\section{世界的物质性及发展规律}

\subsection{世界的多样性与物质统一性}

\subsubsection{物质及其存在方式}

\begin{enumerate}
  \item {\bf 哲学和它的基本问题}
  
  哲学的基本问题是存在和思维的关系问题。它又包括两个方面:
  \begin{itemize}
    \item 物质和精神何者为第一性、何者为第二性的问题
    \begin{itemize}
      \item 这个不是说哪个重要,这个和价值没有关系
    \end{itemize}
    \item 思维和存在有无同一性的问题(思维能否正确认识存在)
  \end{itemize}

  \begin{figure}[h]
    \centering
    \includegraphics[width=0.9\textwidth]{./images/philosophy.png}
    \caption{哲学划分}
  \end{figure}

  \begin{itemize}
    \item 唯物主义总是认同可知论,唯心主义却不总是认同不可知论。
    \item 唯心主义$\neq$形而上学,唯物主义$\neq$辩证法。
    \item 唯心主义又分:
    \begin{itemize}
      \item 主观唯心主义:心、观念、感觉等
      \item 客观唯心主义:理、理念、绝对观念等
    \end{itemize}
  \end{itemize}

  \item {\bf 物质}
  
  \begin{table}[h]
    \centering
    \caption{三种形态的唯物主义对物质的理解}
    \begin{tabular}{|c|c|c|}
      \hline
      形态 & 对物质的理解 & 共同点 \\ \hline
      古代朴素唯物主义 & \makecell[c]{金木水火土等\\ 具体的物质形态} & \multirow{3}{*}{\makecell[c]{从朴素到科学,从片面到比较全面,\\ 但都把世界的本原归结为物质,\\ 主张物质第一性,意识第二性}} \\ \cline{1-2}
      近代形而上学唯物主义 & 原子 & \\ \cline{1-2}
      现代辩证唯物主义 & 客观实在 & \\
      \hline
    \end{tabular}
  \end{table}

  物质的共同特性(唯一特性)是\textbf{客观实在性}。客观实在不一定是看得见摸得着的东西,重点在于它独立于人的意识之外。

  马克思主义物质观的理论意义:
  \begin{itemize}
    \item 略
  \end{itemize}

  \item {\bf 物质的存在方式}
  
  运动是物质的根本属性,物质和运动不可分割。

  运动是绝对的,静止是相对的。

  \item {\bf 物质世界的二重化}
  
  人类改造世界的实践活动使世界发生了二重化,即从自然界中分化出人类社会,从客观世界中分化出主观世界。
\end{enumerate}

\subsubsection{物质和意识的辩证关系}

物质决定意识,意识依赖于物质并反作用于物质。

\begin{enumerate}
  \item {\bf 物质决定意识}
  
  意识的起源:
  \begin{itemize}
    \item 意识是自然界长期发展的产物
    \item 意识又是社会历史发展的产物
    \begin{itemize}
      \item 社会实践特别是劳动在意识的产生和发展中起着决定性的作用。
    \end{itemize}
  \end{itemize}

  意识的本质:意识是特殊的物质——人脑的机能和属性。意识是客观世界的主观映像,是客观内容和主观形式的统一。

  人脑是意识的器官,但不是意识的源泉。意识的源泉是客观世界。

  \item {\bf 意识对物质具有反作用}
  
  表现在:
  \begin{itemize}
    \item 意识活动有目的性和计划性
    \item 意识活动有创造性
    \item 意识有指导实践改造客观世界的作用
    \item 意识具有调控人的行为和生理活动的作用
  \end{itemize}

  \item {\bf 主观能动性和客观规律性的统一}
  
  可出分析题。P17

\end{enumerate}

\subsubsection{世界的物质统一性}

\begin{enumerate}
  \item {\bf 世界的物质统一性原理}
  
  世界是统一的,世界的本原只有一个,就是物质。
  \begin{itemize}
    \item 自然界是物质的。
    \item 人类社会本质上是生产实践基础上形成的物质体系。
    \item 人的意识统一于物质。
  \end{itemize}
  \item {\bf 原理的意义}
  
  \begin{itemize}
    \item {\bf 理论意义}:这原理是辩证唯物主义最基本、最核心的观点,是马克思主义的基石。
    \item {\bf 实践意义}:它是我们从事一切工作的立足点,一切从实际出发是唯物主义一元论的根本要求。
  \end{itemize}

\end{enumerate}

\subsection{事物的普遍联系和变化发展}

\subsubsection{联系和发展的普遍性}

\begin{enumerate}
  \item {\bf 事物的普遍联系}
  
  联系的特点:
  \begin{itemize}
    \item 客观性
    \item 普遍性
    \begin{itemize}
      \item 任何事物内部的不同部分和要素是相互联系的。
      \item 任何事物都不能孤立存在,必然和其他事物相互联系。
      \item 整个世界是相互联系的统一整体。
    \end{itemize}
    \item 多样性
    \begin{itemize}
      \item 直接联系和间接联系
      \item 内部联系和外部联系
      \item 本质联系和非本质联系
      \item 必然联系和偶然联系
    \end{itemize}
    \item 条件性
  \end{itemize}

  \item {\bf 事物的变化发展}
  
  发展的实质是新事物的产生和旧事物的灭亡。

  新生事物是不可战胜的,因为:
  \begin{itemize}
    \item 新事物适应了变化了的环境和条件
    \item 新事物是在旧事物的母体中孕育的
  \end{itemize}

  新旧事物相互区别的根本标志在于是否同历史发展的必然趋势相符合,而不能单凭出现时间的早晚来判断。
\end{enumerate}

\subsubsection{对立统一规律}

\begin{enumerate}
  \item {\bf 唯物辩证法的实质和核心}
  
  对立统一规律,又叫\textbf{矛盾规律},是唯物辩证法的实质和核心,是事物发展的\textbf{根本规律}。

  对立统一规律提供了人们\textbf{认识世界和改造世界的根本方法:矛盾分析方法}。

  是否承认对立统一学说是唯物辩证法和形而上学对立的实质。

  \item {\bf 矛盾的同一性和斗争性及其在事物发展中的作用}
  
  同一性和斗争性是矛盾的两种\textbf{基本属性}。

  什么是同一性:
  \begin{itemize}
    \item 矛盾的对立面相互依存,互为存在。
    \item 矛盾的对立面在一定条件下可以相互转化。
  \end{itemize}

  \begin{itemize}
    \item 矛盾的斗争性是无条件的、绝对的
    \item 矛盾的同一性是有条件的、相对的
  \end{itemize}

  同一性在事物发展中的作用:
  \begin{itemize}
    \item 矛盾双方可以利用对方的发展使自己获得发展。
    \item 矛盾双方可以相互吸取有利于自身的因素,在相互作用中各自得到发展。
    \item 矛盾双方可以向着自己的对立面转化而得到发展,并规定着事物发展的方向。
  \end{itemize}

  斗争性在事物发展中的作用:
  \begin{itemize}
    \item 促进矛盾双方力量的变化,造成事物的量变,为对立面的转化、事物的质变创造条件。
    \item 斗争促使矛盾双方地位或性质转化,实现事物的质变。
  \end{itemize}

  矛盾的斗争性和同一性共同作用,推动了事物的发展。在不同条件下,二者所处的地位可能不同。

  方法论意义:
  \begin{itemize}
    \item 在分析和解决矛盾时,必须从对立中把握同一,从同一中把握对立。
    \item 正确把握\textbf{和谐}对事物发展的作用。
    \begin{itemize}
      \item \textbf{和谐}是矛盾的一种特殊表现形式,体现着矛盾双方的相互依存、相互促进、共同发展。和谐是相对的、有条件的。
    \end{itemize}
  \end{itemize}

  事物发展的根本原因不在事物外部,而在事物内部的矛盾性。

  \item {\bf 矛盾的普遍性和特殊性及其相互关系}
  
  \begin{itemize}
    \item {\bf 普遍性}:是矛盾的共性。即:一切事物都是对立统一的,矛盾是事物的普遍本质。
    \item {\bf 特殊性}:是矛盾的个性。
    \begin{itemize}
      \item 不同事物的矛盾各有其特点
      \item 同一事物的矛盾在不同发展过程和不同发展阶段各有其特点
      \item 构成事物的\textbf{诸多矛盾}以及每一矛盾的\textbf{不同方面}各有不同的性质、地位和作用。
      \begin{itemize}
        \item 事物由多种矛盾构成,可分为主要矛盾和次要矛盾。
        \item 每种矛盾又有主要方面和次要方面。
        \item 事物的性质由主要矛盾的主要方面规定。
        \item 在实际工作中,要坚持“两点论”和“重点论”的统一。
      \end{itemize}
    \end{itemize}
  \end{itemize}

  可以说共性寓于个性之中,但不能说个性寓于共性之中,因为个性比共性更丰富。

  矛盾的普遍性和特殊性的辩证关系原理是马克思主义的普遍整理同各国的具体实际相结合的哲学基础。
\end{enumerate}

\subsubsection{量变质变规律和否定之否定规律}

\begin{enumerate}
  \item {\bf 量变质变规律}
  
  事物的质、量、度:
  \begin{itemize}
    \item {\bf 质}:是事物成为自身并区别于其他事物的内在规定性。
    \item {\bf 量}:是事物的规模、程度、速度等可以用数量关系表示的规定性。
    \item {\bf 度}:是保持事物的稳定性的数量界限。超出度的范围,此物就转化为他物。
  \end{itemize}

  区分量变和质变的\textbf{根本标志}是事物的变化是否超出度。

  量变和质变的辩证关系:
  \begin{itemize}
    \item 量变是质变的重要准备
    \item 质变是量变的必然结果
    \item 量变和质变相互渗透
  \end{itemize}

  量变质变规律体现了事物发展的渐进性和飞跃性的统一。

  \item {\bf 否定之否定规律}
  
  任何事物内部都有肯定因素和否定因素。
  \begin{itemize}
    \item 肯定因素是维持现存事物存在的因素
    \item 否定因素是促使现存事物灭亡的因素
  \end{itemize}
\end{enumerate}

\end{document}